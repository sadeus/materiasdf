\documentclass[a4paper]{article}
\usepackage[utf8]{inputenc}
\usepackage[spanish]{babel}
\usepackage{amsmath,amssymb,amsfonts} %Fonts de AMS
\usepackage{mathbbol} %Font para matrices como la identidad
%%Formato de hoja%%
\usepackage[margin=1.5cm]{geometry}

%%Imagenes%%
\usepackage{graphicx}
\usepackage{wrapfig}
\usepackage{caption}
\usepackage{subcaption}
\graphicspath{{./fig/}}
\usepackage{float}

%%Ecuaciones y teoremas
\numberwithin{equation}{section} %Número de ecuación por sección

%\newtheorem{definition}{Definición}[chapter]
%\newtheorem{axiom}{Axioma}[chapter]
%\newtheorem{law}{Ley}[chapter]
%\newtheorem{principle}{Principio}[chapter]
%\newtheorem{postulate}{Postulado}[chapter]
%\newtheorem{collorary}{Colorario}[chapter]
%\setcounter{\thelaw}{0}
\renewcommand{\vec}[1]{\boldsymbol{#1}}
\newcommand{\bb}[1]{\boldsymbol{#1}}
\newcommand{\rinv}{\frac{1}{|\vec{r} - \vec{r}'|}}
\newcommand{\Rinv}{\frac{\vec{r} - \vec{r}'}{|\vec{r} - \vec{r}'|^3}}

\title{Notas de Física Teórica 1}
\author{S. Schiavinato}
\date{}

\begin{document}
\maketitle
\tableofcontents



\section{Electroestática}

\begin{equation}
\vec{F}_{1\to2} = q_1 q_2\frac{\vec{r}_2 - \vec{r}_1}{|\vec{r}_2 - \vec{r}_1|^3} = - \vec{F}_{2\to1}
\label{eq:electroestatica_coulomb}
\end{equation}

Definimos el campo eléctrico con una carga de prueba $q$ (que no deforma la configuración de cargas presente) como
\begin{equation}
\vec{E} = \frac{\vec{F}}{q}
\label{eq:campo_electrico}
\end{equation}
Experimentalmente vale el principio de superposición, por lo que el campo eléctrico general de una distribución de carga con densidad de carga $\rho(\vec{r}')$ es
\begin{equation}
\vec{E}(\vec{r}) = \int \rho(\vec{r}') \frac{\vec{r} - \vec{r}'}{|\vec{r} - \vec{r}'|^3} d\vec{r}'
\label{eq:campo_electrico_dist_carga}
\end{equation}
Que se transforma en la expresión para una carga con la siguiente distribución
\begin{equation}
 \rho(\vec{r}) = q \delta(\vec{r} - \vec{r}')
\end{equation}
que es la delta de Dirac. Esta distribución tiene la particularidad de
\begin{equation}
\nabla \cdot \left(\frac{\vec{r} - \vec{r}'}{|\vec{r} - \vec{r}'|^3}\right) = 4\pi\delta(\vec{r} - \vec{r}')
\end{equation}
y además sabemos que
\begin{equation}
\nabla \left(\frac{1}{|\vec{r} - \vec{r}'|}\right) = -\frac{\vec{r} - \vec{r}'}{|\vec{r} - \vec{r}'|^3}
\end{equation}
Con estas dos expresiones podemos calcular la divergencia del campo eléctrico
\[ \nabla \cdot \vec{E}(\vec{r}) = \nabla \cdot \int \rho(\vec{r}') \frac{\vec{r} - \vec{r}'}{|\vec{r} - \vec{r}'|^3} d\vec{r}' = \int \rho(\vec{r}') \, \nabla \cdot \frac{\vec{r} - \vec{r}'}{|\vec{r} - \vec{r}'|^3} d\vec{r}' = 4\pi \int \rho(\vec{r}') \delta(\vec{r} - \vec{r}') d\vec{r}'\]
nos que nos da
\begin{equation}
\nabla \cdot \vec{E}(\vec{r}) = 4\pi \rho(\vec{r})
\label{eq:gauss_diff}
\end{equation}
y el rotor del campo eléctrico es
\[\nabla \times \vec{E}(\vec{r}) = \nabla \times \int \rho(\vec{r}') \frac{\vec{r} - \vec{r}'}{|\vec{r} - \vec{r}'|^3} d\vec{r}' = \int \rho(\vec{r}') \nabla \times \frac{\vec{r} - \vec{r}'}{|\vec{r} - \vec{r}'|^3} d\vec{r}' = - \int \rho(\vec{r}') \nabla \times \nabla \left(\frac{1}{|\vec{r} - \vec{r}'|}\right) d\vec{r}'\]
lo que nos deja
\begin{equation}
\nabla \times \vec{E}(\vec{r}) = 0
\label{eq:electroestatica_rotor}
\end{equation}
Con esas ecuaciones podemos encontrar el campo eléctrico para cualquier configuración.

Las ecuaciones diferenciales del campo eléctrico tienen asociadas ecuaciones de continuidad, que se obtienen observando de forma cercana una interfaz
\begin{equation}
 (\vec{E}_2 - \vec{E}_1) \cdot \vec{n} = 4 \pi \sigma \qquad \vec{n} \times (\vec{E}_2 - \vec{E}_1) = 0 
 \label{eq:electroestatica_continuidad_campo}
\end{equation}
donde queda explicito que la componente que es discontinua es la normal y la tangencial debe ser continua. 


Para hacerlo más fácil sabemos que el campo eléctrico debe ser
\begin{equation}
 \vec{E}(\vec{r}) = - \nabla \varphi(\vec{r})
 \label{eq:electroestatico_potencial}
\end{equation}
lo que nos define
\begin{equation}
 \varphi(\vec{r}) = - \int_C \vec{E} \cdot d\vec{l}
 \label{eq:electroestatico_potencial_integral}
\end{equation}
con lo que la divergencia del campo nos da
\begin{equation}
 \nabla^2 \varphi(\vec{r}) = - 4  \pi \rho(\vec{r})
 \label{eq:electroestatico_potencial_poisson}
\end{equation}
Que corresponde a la ecuación de Poisson cuando hay carga libre o a la ecuación de Laplace
\begin{equation}
 \nabla^2 \varphi(\vec{r}) = 0
 \label{eq:electroestatico_potencial_laplace}
 \end{equation}
 si no hay carga libre. A partir de la fuerza de Coulomb y con la definición del potencial tenemos que se verifica
\begin{equation}
	\varphi(\vec{r}) = \int \frac{\rho(\vec{r}'}{|\vec{r} - \vec{r}'|} d\vec{r}'
	\label{eq:electroestatica_potencial_superposicion_integral}
\end{equation}
si y solo si no hay contornos, para lo que vamos a utilzar otras herramientas.

 La ecuaciones de continuidad para el potencial corresponde a reemplazar en la ecuación \ref{eq:electroestatica_continuidad_campo} el potencial, obteniendo
 \begin{equation}
  \phi_1 = \phi_2 \qquad \frac{\partial \phi_1}{\partial n} - \frac{\partial \phi_2}{\partial n} = 4 \pi \sigma
 \end{equation}
 
\section{Magnetoestática}
Un campo magnético, que puede tener la fuente presente o no en el entorno del problema, produce una fuerza sobre una carga de prueba $q$ igual a
\begin{equation}
F = q \frac{\vec{v}}{c} \times \vec{B}
\end{equation}
donde el producto $\times$ corresponde al producto vectorial. Esto lo tomamos como la definición del campo magnético, para tener una base mecánica sobre sustentar la teoría.
Experimentalmente, se observa la siguiente ley, Biot-Savart, relacionando el movimiento de cargas o \emph{corrientes} por hilos conductores y el campo magnético generado.
\begin{equation}
 d\vec{B} = \frac{1}{c} \frac{I d\vec{l} \times \vec{r}}{r^3}
\end{equation}
es decir
\begin{equation}
    \vec{B} = \frac{1}{c} \int  \frac{I d\vec{l} \times \vec{r}}{r^3}.
\end{equation}
Esto se generaliza a corrientes volumétricas de la siguiente forma
\begin{equation}
    \vec{B} = \frac{1}{c} \int  \vec{J}(\vec{r}') \times \Rinv d\vec{r}'
    \label{eq:biot_savart}
\end{equation}

De la misma forma, vamos a buscar expresiones para el rotor y la divergencia de campo magnético. Primero reescribamos un poco la expresión general de la ley de Biot-Savart (eq \ref{eq:biot_savart}), usando la regla del producto del rotor
\[ \nabla \times (\phi \vec{A}) = (\nabla \phi) \times \vec{A} + \phi \nabla \times \vec{A} \]
y que $\vec{J}$ no depende de la coordenada campo $\vec{r}$.
\[ \vec{B} = \frac{1}{c} \int  \vec{J}(\vec{r}') \times \left(- \nabla\left(\rinv\right) \right) d\vec{r}' = \frac{1}{c} \int  \nabla\left(\rinv\right) \times \vec{J}(\vec{r}') d\vec{r}' = \frac{1}{c} \int \nabla \times \left( \frac{\vec{J}(\vec{r}')}{|\vec{r} - \vec{r}'|}\right) d\vec{r}'\]
Como la derivada no depende del integrando, las podemos intercambiar sin problemas, pero a nosotros nos interesa la divergencia de esto
\[ \nabla \cdot \vec{B} = \frac{1}{c} \int  \nabla \cdot \nabla \times \left( \frac{\vec{J}(\vec{r}')}{|\vec{r} - \vec{r}'|}\right) d\vec{r}' = 0\]
es decir
\begin{equation}
    \nabla \cdot \vec{B} = 0
    \label{eq:gauss_magnetico}
\end{equation}
Para el rotor del campo magnético usamos la siguiente identidad
\begin{equation}
\nabla \times \left( \nabla \times \vec{A} \right) = \nabla(\nabla \cdot \vec{A}) - \nabla^{2}\vec{A}
\end{equation}
que nos queda
\[ \nabla \times \vec{B} = \frac{1}{c} \int  \nabla \times \nabla \times \left( \frac{\vec{J}(\vec{r}')}{|\vec{r} - \vec{r}'|}\right) d\vec{r}' = \frac{1}{c} \int \left\{ \nabla  \left(\vec{J}(\vec{r}') \cdot \nabla \left( \rinv \right) \right) - \vec{J}(\vec{r}') \nabla^2 \left( \rinv \right) \right\} d\vec{r}' \]

Para el segundo término usamos que
\begin{equation}
    \nabla^2 \phi = \nabla \cdot (\nabla \phi)
\end{equation}
es decir
\begin{equation}
    \nabla^2 \left(\rinv\right) = - 4\pi \delta(\vec{r} - \vec{r}')
\end{equation}
y para el primero integramos por partes
\[ \nabla \times \vec{B} = \frac{1}{c} \nabla \int \frac{\nabla \cdot \vec{J}}{|\vec{r} - \vec{r}'|} d\vec{r}' + \frac{4\pi}{c} \vec{J}(\vec{r})\]
y como estamos en magnetoestática, sabemos que
\begin{equation}
\nabla \cdot \vec{J} = 0
\end{equation}
por lo que nos queda la ley de Ampere, en forma diferencial,
\begin{equation}
\nabla \times \vec{B} = \frac{4\pi}{c} \vec{J}(\vec{r})
\label{eq:ampere}
\end{equation}

De la misma forma que para el campo eléctrico, la continuidad del campo magnético estático corresponde a 
\begin{equation}
	(\vec{B}_2 - \vec{B}_1)\cdot \vec{n} = 0 \qquad  \vec{n} \times (\vec{B}_2 - \vec{B}_1) = \frac{4\pi}{c} \vec{K}  
\end{equation}
donde $\vec{K}$ corresponde a la corriente superficial, tal que tiene solo componentes en el plano de la interfaz.

Ahora, como el campo magnético $\vec{B}$ es un campo sin divergencia, podemos defir otro campo, llamado \emph{potencial vector magnético} o potencial vector, tal que
\begin{equation}
\vec{B} = \nabla \times \vec{A}
\end{equation}
con el que obtengo la siguiente ecuación diferencial (a partir de la ley de Ampere)
\begin{equation}
\nabla^2 \vec{A} - \nabla (\nabla \cdot \vec{A}) = - \frac{4\pi}{c} \vec{J}
\end{equation}
que se puede simplificar, sabiendo que podemos elegir diferentes potenciales a menos de un gradiente (que se anula al calcularle el rotor)
\begin{equation}
\vec{A} \to \vec{A} + \nabla \varphi
\label{eq:gauge_A}
\end{equation}
con lo que la solución a la ecuación diferencial finalmente nos queda
\begin{equation}
\vec{A} = \frac{1}{c} \int  \frac{\vec{J}(\vec{r}')}{|\vec{r} - \vec{r}'|} d\vec{r}' + \nabla \varphi
\end{equation}
y si elegimos $\varphi$ tal que
\begin{equation}
\nabla \cdot \vec{A} = 0
\label{eq:gauge_coulumb}
\end{equation}
que se denomina \emph{gauge de Coulomb}, tenemos
\begin{equation}
\nabla^2 \vec{A} = - \frac{4\pi}{c} \vec{J}
\end{equation}
que se soluciona con
\begin{equation}
\vec{A} = \frac{1}{c} \int  \frac{\vec{J}(\vec{r}')}{|\vec{r} - \vec{r}'|} d\vec{r}'
\end{equation}

\section{Simetrías}
Con la fuerza de Lorentz
\begin{equation}
\vec{F} = q \left(\vec{E} + \frac{\vec{v}}{c} \times \vec{B}\right)
\end{equation}
aparecen ciertas simetrías, propias de exigir la homogeneidad e isotropía del espacio; es decir, si rotamos o trasladamos las fuentes y las cargas de prueba las fuerzas deben ser iguales, impactando en las fuentes.
Transladar las fuentes implica
\begin{equation}
\rho'(\vec{r}) = \rho(\vec{r} - \vec{a}) \quad \vec{j}'(\vec{r}) = \vec{j}(\vec{r} - \vec{a})
\end{equation}
e implica que las fuerzas
\[q \left(\vec{E}'(\vec{r}) + \frac{\vec{v}}{c} \times \vec{B}'(\vec{r})\right) = q \left(\vec{E}(\vec{r} - \vec{a}) + \frac{\vec{v}}{c} \times \vec{B}(\vec{r} - \vec{a})\right)\]
por lo que nos queda (considerando que las cargas de prueba no tienen diferente velocidad)
\begin{equation}
\vec{E}'(\vec{r}) = \vec{E}(\vec{r} - \vec{a}) \quad \vec{B}'(\vec{r}) = \vec{B}(\vec{r} - \vec{a})
\end{equation}

Mientras, rotar las fuentes implica
\begin{equation}
\rho'(\vec{r}) = \rho(\vec{R}^{-1} \vec{r}) \qquad \vec{j}'(\vec{r}) = \vec{R}\vec{j}(\vec{R}^{-1} \vec{r})
\end{equation}
donde la matriz u operador $\vec{R}$ es una matriz ortogonal que representa la translación.

Con esta expresión, es evidente que el campo eléctrico se translada como
\begin{equation}
\vec{E}'(\vec{r}) = \vec{R} \vec{E}(\vec{R}^{-1} \vec{r})
\end{equation}
mientras que para el campo magnético debemos usar la siguiente identidad
\begin{equation}
\vec{R} (\vec{A} \times \vec{B}) = (\vec{R} \vec{A}) \times (\vec{R} \vec{B})
\end{equation}
que nos termina dando que el campo magnético rota con las fuentes, salvo un factor, es decir
\begin{equation}
\vec{B}'(\vec{r}) = \det(\vec{R}) \vec{B}(\vec{R}^{-1} \vec{r})
\end{equation}
El factor $\det(\vec{R})$ proviene de considerar las reflexiones, que tienen determinante negativo, y por lo tanto cambian el signo del campo magnético; el campo magnético es un \textbf{pseudo vector}.

Estas simetrías corresponden a las atadas a la realidad mecánica, pero no nos dice que las ecuaciones que relacionan las fuentes con los campos (las ecuaciones de la electrodinámica o de Maxwell) verifican estas simetrías. Usando notación de indices, %HACER
para las ecuaciones de la electroestática y magnetoestática, se verifica las simetrías de traslación y rotación ya predichas; es más podemos agregar la componente temporal, y trabajar con la electrodinámica y obtendríamos el mismo resultado.

\section{Métodos de resolución}
En esta sección vamos a tratar de resolver las ecuaciónes del potencial electroestático, las ecuaciones \ref{eq:electroestatico_potencial_poisson} y \ref{eq:electroestatico_potencial_laplace}, con diferentes condiciones de contorno y cargas; la idea de resolver la ecuación del potencial es no tener que resolver cada componente del campo, aunque siempre va a haber una \emph{libertad} de definición (o \emph{gauge}). Recordemos, la ecuación que vamos a resolver es:
\[ \nabla^2 \varphi(\vec{r}) = - 4 \pi \rho(\vec{r}) \]
Además, las herramientas que implementemos acá nos servirán para los problemas de magnetoestática.

\subsection{Separación de variables}
Uno de los métodos más usuales para resolver ecuaciones homogeneas (sin fuentes), cuando las condiciones de contorno son \emph{razonables} (o mejor dicho, \emph{separables}), es el de \textbf{separación de variables}. Este consiste, en grandes rasgos, proponer lo siguiente
\begin{equation}
  \varphi(x_1, x_2, x_3) = \varphi_{1}(x_1) \varphi_{2}(x_2) \varphi_{3}(x_3)
  \label{eq:separacion_variables_multiplicacion}
\end{equation}
Con esto, la ecuación de Laplace nos queda
\begin{equation}
  \nabla^2 \varphi =  \varphi_2  \varphi_3 \nabla_{x_1}^2 \varphi_1 +  \varphi_1  \varphi_3 \nabla_{x_2}^2 \varphi_2 +   \varphi_1  \varphi_2 \nabla_{x_3}^2 \varphi_3 = 0
\end{equation}
y por lo tanto nos queda
\begin{equation}
   \frac{1}{\varphi_{x_1}} \nabla_{x_1}^2 \varphi_1 +   \frac{1}{\varphi_{x_2}} \nabla_{x_2}^2 \varphi_2 + \frac{1}{\varphi_{x_3}} \nabla_{x_3}^2 \varphi_3 = 0
\end{equation}
Así nos quedó una suma de tres términos, cada uno con una relación funcional de una sola variable; sin las condiciones de contorno no mezclan coordenadas, permitirán definir la siguiente terna de ecuaciones
\begin{equation}
 \nabla_i^2 \varphi_i = \lambda_i \varphi_i \qquad \sum^3_{i = 1} \lambda_i = 0
 \label{eq:separacion_variables_ecuacion}
\end{equation}
Dejamos expresado el operador laplaciano porque solamente en coordenadas cartesianas es simplemente suma de derivadas segundas. Ahora, dada las ecuaciones, pasamos a considerar algunos tipos de coordenadas, donde el operador laplaciano es \emph{separable}. En estos casos, el problema se reduce a lo que se llama \emph{problema de Stum-Lioville}, que tienen solución usando herramientas del álgebra lineal.

\subsubsection{Coordenadas cartesianas}
En coordenadas cartesianas, tenemos que la ecuación \ref{eq:separacion_variables_ecuacion} se escribe
\begin{equation}
  \frac{\partial^2}{\partial x_i^2} \varphi_{i} = \lambda_i \varphi_i \qquad \lambda_x + \lambda_y + \lambda z = 0
\end{equation}
Consideremos el potencial dentro de una caja cubica de lado $L$ (para simplificar la exposición), con todas los extremos a potencial nulo salvo la cara de $z = L$, que tiene un potencial $V(x,y)$. De esta forma elegimos la siguiente propiedad para las constantes $\lambda_i$:
\begin{equation}
  \lambda_x = - \alpha^2 \quad \lambda_y = - \beta^2 \quad \lambda_z = \gamma^2 \quad \Rightarrow \quad \alpha^2 + \beta^2 = \gamma^2
\end{equation}
que produce el siguiente conjunto de ecuaciones diferenciales
\begin{equation}
  \begin{gathered}
    \partial_{xx} \varphi_x = - \alpha{^2} \varphi_x \\
    \partial_{yy} \varphi_y = - \beta{^2} \varphi_y \\
    \partial_{zz} \varphi_z =  \gamma{^2} \varphi_z
  \end{gathered}
\end{equation}
que tienen como solución
\begin{equation}
  \begin{gathered}
    \varphi_{x} = e^{\pm i \alpha x}\\
    \varphi_y = e^{\pm i \beta y}\\
    \varphi_z = e^{\pm \sqrt{\alpha^2 + \beta^2} z}
  \end{gathered}
  \label{eq:separacion_variables_cartesiana}
\end{equation}
Esta solución es bastante general y vamos a utilizarla en la mayoría de los casos: por superposición y renombramiento de variables podemos generar cualquier elección de contorno.

Si ahora le imponemos las condiciones que consideramos, tenemos que
\begin{equation}
  \begin{gathered}
    \varphi_{x} = \sen(\alpha_n x)\\
    \varphi_y = \sen(\beta_m x)\\
    \varphi_z = \senh(\sqrt{\alpha_n^2 + \beta_m^2} z)\\
    \alpha_n = \frac{n \pi}{L} \quad \beta_m = \frac{m \pi}{L}
  \end{gathered}
\end{equation}
por lo que la solución final es una combinación lineal de cada solución
\begin{equation}
  \varphi(x, y, z) = \sum^\infty_{n, m = 1} A_{nm} \sen(\alpha_n x)\; \sen(\beta_m x)\; \senh(\sqrt{\alpha_n^2 + \beta_m^2} z)
\end{equation}
que finalmente nos permite encontrar las componentes $A_{nm}$ a partir del potencial $V(x,y)$ como
\begin{equation}
  V(x, y, z) = \sum^2_{n, m = 1} A_{nm} \sen(\alpha_n x)\; \sen(\beta_m x)\; \senh(\sqrt{\alpha_n^2 + \beta_m^2} L)
\end{equation}
y sabiendo que las funciones senoidales son funciones ortogonales y forman una base, es decir
\begin{equation}
  \int_{0}^{L} \int_{0}^{L} \sen\left(\frac{\pi n}{L} x\right) \sen(\frac{\pi m}{L} y) dx dy = \frac{2}{L} \delta_{mn}
\end{equation}
podemos escribir a las componentes $A_{mn}$ como
\begin{equation}
  A_{mn} = \frac{4}{L^2 \senh(\gamma_{mn} L)}\int_{0}^{L} \int_{0}^{L} V(x, y) \sen(\alpha_n x) \sen(\beta_m y) dx dy
\end{equation}

Acá salta a la luz el mecanismo usual de resolver un problema por separación de variables:
\begin{itemize}
  \item{Proponer coordenadas que permitan separar la ecuación de Laplace, ecuación \ref{eq:electroestatico_potencial_laplace}, y las condiciones de contorno. Aparecen tres ecuaciones diferenciales ordinarias}
  \item{Dos de estas ecuaciones se resuelven como un problema de Stum-Lioville, que asegura que existe un conjunto ortonormal de funciones asociado a un conjunto de autovalores. Las coordenadas que van a ser base tienen condiciones de contorno homogeneas (es decir nulas) o no están acotadas}
  \item{Aplicamos condiciones de contorno y encontramos el valor de los autovalores y de las componentes libres}
\end{itemize}
Si es necesario, se pueden "pegar" dos soluciones, exigiendo la continuidad del potencial (debido a que sabemos que el campo eléctrico es proporcional a la fuerza y por lo tanto debe tener integral continua) y encontrando las constantes libres. 



Si en una dirección no se tiene contorno, es decir la variable recorre todo el eje o un semieje, las soluciones posibles son exponenciales imaginarias. Por ejemplo, tenemos que en el eje $x$ no hay contorno, por lo tanto la solución es
\begin{equation}
 X_k(x) = e^{i k x} \qquad -\infty < k \infty
\end{equation}
con la siguiente relación de ortogonalidad
\begin{equation}
 \frac{1}{2\pi} \int_{-\infty}^{\infty} X_k(x) X_{k'}(x) dx = \frac{1}{2\pi} \int_{-\infty}^{\infty} e^{i (k - k') x} dx = \delta(k - k') 
\end{equation}
Sabiendo que son base, cualquier función $f(x)$ en el intervalo real puede ser descripta por su equivalente en el $k$-espacio como
\begin{equation}
f(x) = \frac{1}{\sqrt{2\pi}} \int_{-\infty}^{\infty} \acute{F}(k) e^{i k x} dx \qquad \acute{F}(k) = \frac{1}{\sqrt{2\pi}} \int_{-\infty}^{\infty} f(x) e^{-i k x} dx
\end{equation}

De esta forma el potencial queda definido como
\begin{equation}
\phi(\vec{r}) = \int_{-\infty}^{\infty} e^{i k x} Y(y) Z(z) dx
\end{equation}
donde tenemos otras coordenadas no especificadas para destacar la generalidad del resultado.


\subsubsection{Coordenadas esféricas}
Cuando se trabaja con coordenadas esféricas, con $r$ la coordenadas radial, $\theta$ el ángulo azimutal y $\phi$ el ángulo polar, se puede demostrar que la ecuación de Laplace (y por analogía la de Poisson) queda
\begin{equation}
\frac{1}{r} \frac{\partial^2}{\partial r^2} (r \varphi) + \frac{1}{r^2 \sen(\theta)} \frac{\partial}{\partial \theta} \left( \sen(\theta) \frac{\partial \varphi}{\partial \theta}\right) + \frac{1}{r^2 \sen^2(\theta)} \frac{\partial^2 \varphi}{\partial \phi^2}  = 0
\label{eq:electroestatica_laplace_esfericas}
\end{equation}
Si se propen la solución
\begin{equation}
\varphi = \frac{U(r)}{r} P(\theta) Q(\phi)
\end{equation}
y multiplicamos por $r^2 \sen(\theta) / U P Q$ tenemos
\begin{equation}
r^2 \sen^2(\theta) \left[ \frac{1}{U} \frac{d^2 U}{d r^2} + \frac{1}{P r^2 \sen(\theta)} \frac{d}{d\theta} \left( \sen(\theta) \frac{d P}{d\theta}\right)\right] + \frac{1}{Q} \frac{d^2 Q}{d \phi^2} = 0
\end{equation}

Esta ecuación usualmente se separa inicialmente por la parte polar, eligiendo
\begin{equation}
 \frac{1}{Q} \frac{d^2 Q}{d \phi^2} = -m^2
\end{equation}
por lo que el resto es
\[ r^2 \sen^2(\theta) \left[ \frac{1}{U} \frac{d^2 U}{d r^2} + \frac{1}{P r^2 \sen(\theta)} \frac{d}{d\theta} \left( \sen(\theta) \frac{d P}{d\theta}\right)\right] - m^2 = 0 \]
La próxima separación de variables que se efectua consiste en considerar que
\begin{equation}
\frac{1}{U} \frac{d^2 U}{\partial r^2} = l (l + 1) 
\end{equation}
por lo que llegamos a que la ecuación azimutal queda
\begin{equation}
\frac{1}{\sen(\theta)} \frac{d}{d\theta} \left( \sen(\theta) \frac{d P}{d\theta}\right) + \left[l (l + 1) \frac{m^2}{\sen^2(\theta)}\right] P = 0
\end{equation}

Esta elección de separación de variables no es casual ya si elegimos $x = \cos(\theta)$ (y hacemos el cambio de variables) tenemos la ecuación de Legrendre generalizada
\begin{equation}
\frac{d}{d\theta} \left( (1 - x^2) \frac{d P}{d x}\right) + \left[l (l + 1) \frac{m^2}{1 - x^2}\right] P = 0
\end{equation}
que tiene como solución los polinomios de Legrendre asociados $P^{m}_l(x)$ que los podemos obtener con la formula de Rodriguez
\begin{equation}
P^{m}_l = \frac{(-1)^m}{2^l l!} (1 - x^2)^{m/2} \frac{d^{l + m}}{d x^{l + m}} (x^2 - 1)^l
\end{equation}
con los indices libres con la siguientes propiedades
\begin{equation}
l \in \mathbb{N} \qquad -l < m < l 
\end{equation}

Estos polinomios tienen, además, la siguiente relación de ortogonalidad
\begin{equation}
\int_{-1}^{1} P^m_{l}(x) P^m_{l'}(x) dx = \frac{2}{2l + 1} \frac{(l + m)!}{(l - m)!} \delta_{l \; l'}
\end{equation}



Con la solución polar, que corresponde a 
\begin{equation}
Q(\phi) = e^{i m \phi}
\end{equation}
podemos construir los llamados \emph{armónicos esféricos}, que son a su vez ortogonales
\begin{equation}
Y_{lm}(\theta,\phi) = P(\theta) Q(\phi) = \sqrt{\frac{2l + 1}{4\pi} \frac{(l + m)!}{(l - m)!}} P^{m}_l(\cos(\theta)) e^{i m \phi}
\end{equation}
que tienen la relación de normalización
\begin{equation}
\int_{0}^{2\pi} \int_{0}^{\pi} Y^{*}_{l' \; m'} Y_{l \; m} \sen(\theta) d\theta d\phi = \delta_{l\;l'} \delta_{m'\; m}
\end{equation}

Si el problema tiene simetría axial, es decir no depende de $\phi$, la única forma de eliminar la dependencia de esta variable es exigir $m = 0$. 

Mientras, la solución radial corresponde a
\begin{equation}
R(r) = A_{l m} r^{l + 1} + B_{l m} r^{-l}
\end{equation}
donde disponemos las constantes, propias de la condiciones de contorno, en la parte radial. Esto siempre es posible de hacer, no implica ninguna restricción en los contornos elegidos, salvo que estos sean separables en radios fijos; de esta forma no tiene sentido hablar de una solución no acotada, ya que el infinito aparece naturalmente en la parte radial (exiguiendo eliminar la solución divergente).

Finalmente el potencial en esféricas queda
\begin{equation}
\varphi(r, \theta, \phi) = \sum_{l, m} [A_{l m} r^{l + 1} + B_{l m} r^{-l}] Y_{l m}(\phi, \theta) =  \sum_{l, m} \sqrt{\frac{2l + 1}{4\pi} \frac{(l + m)!}{(l - m)!}}  [A_{l m} r^{l + 1} + B_{l m} r^{-l}] P^{m}_l(\cos(\theta)) e^{i m \phi}
\end{equation}

\subsubsection{Coordenadas cilindricas}
En coordenadas cilindricas, $\rho$, $\phi$ y $z$, la ecuación de Laplace queda
\begin{equation}
	\frac{\partial^2 \varphi}{\partial \rho^2} + \frac{1}{\rho} \frac{\partial \varphi}{\partial \rho} + \frac{1}{\rho^2} \frac{\partial^2 \varphi}{\partial \phi^2} + \frac{\partial^2 \varphi}{\partial z^2} = 0
	\label{eq:electroestatica_separacion_cilindricas_laplace}
\end{equation}

Eligiendo la solución usual
\begin{equation}
	\varphi(\rho, \phi, z) = R(\rho) Q(\phi) Z(z)
\end{equation}
tenemos las siguientes ecuaciones diferenciales
\begin{equation}
	\begin{gathered}
		\frac{\partial^2 Z}{\partial z^2} - k^2 Z = 0 \\
		\frac{\partial^2 Q}{\partial \phi^2} + \nu^2 Q = 0 \\
		\frac{\partial^2 R}{\partial \rho^2} + \frac{1}{\rho} \frac{\partial R}{\partial \rho} + \left(k^2 - \frac{\nu^2}{\rho^2} \right) R = 0
	\end{gathered}
\end{equation}

La parte radial la resolvemos con un cambio de variable $x = k \rho$, a lo que llegamos a
\begin{equation}
		\frac{\partial^2 R}{\partial x^2} + \frac{1}{x} \frac{\partial R}{\partial x} + \left(1 - \frac{\nu^2}{x^2} \right) R = 0
\end{equation}
que tiene como solución los polinimios de Bessel.

\subsection{Funciones de Green}
Sabemos que la solución a la ecuación de Poisson no acotada (cotorno en el infinito) es la ecuación
\begin{equation}
	\varphi(\vec{r}) = \int \frac{\rho(\vec{r}'}{|\vec{r} - \vec{r}'|} d\vec{r}'
\end{equation}
Si tenemos un contorno, podemos encontrar la solución por imágenes, aunque puede resultar complejo analíticamente. Otro método consiste en encontrar las funciones de Green, para lo que primero debemos encontrar las condiciones de contorno.

El teorema de Gauss o de la divergencia asegura que
\begin{equation}
	\int_{V} \nabla \cdot \vec{A} d\vec{r} = \oint_{\partial V} \vec{A} \cdot \vec{n} dS
\end{equation}
es decir el flujo de un campo por la superficie de un volumen cerrado es igual a la integral volumetrica de la divergencia en ese volumen. Si elegimos $\vec{A} = \phi \nabla \psi$, siendo estas funciones continuas, tenemos que el teorema de la divergencia nos queda
\[ \int_{V} (\phi \nabla^2 \psi + \nabla \phi \cdot \nabla \psi) d\vec{r} = \oint_{\partial V} \phi \nabla \psi \cdot \vec{n} dS \]
Ahora nos eliminamos del término $\nabla \phi \cdot \nabla \psi$ restandole exactamente la misma expresión intercambiando $\psi$ con $\phi$, con lo que nos queda
\begin{equation}
	\int_{V} (\phi \nabla^2 \psi - \psi \nabla^2 \phi) d\vec{r} = \oint_{\partial V} (\phi \nabla \psi \cdot \vec{n} - \psi \nabla \phi \cdot \vec{n}) dS
\end{equation}
que corresponde a la primera identidad de Green. Si elegimos $\phi = \varphi$, el potencial eléctrico, y $\psi = \frac{1}{|\vec{r} - \vec{r}'|}$, y reemplazamos los laplacianos conocidos tenemos que
\[ \int_{V} \left[ - 4\pi varphi(\vec{r}') \delta(\vec{r} - \vec{r}') + \frac{4\pi \rho(\vec{r}'}{|\vec{r} - \vec{r}'} \right] d\vec{r}' = \oint_{\partial V} \left[\varphi \nabla \left(\frac{1}{|\vec{r} - \vec{r}'|}\right) - \rinv \nabla \varphi \cdot \vec{n} \right] dS \]
Si hacemos la integral en todo el espacio, esto nos queda
\begin{equation}
	\vaphi(\vec{r}) = \int_{V} \frac{\rho(\vec{r}')}{|\vec{r} - \vec{r}'|} d\vec{r}' + \frac{1}{4\pi} \oint_{\partial V} \left[\rinv \nabla \varphi \cdot \vec{n} - \varphi \nabla \left(\frac{1}{|\vec{r} - \vec{r}'|}\right)  \right] dS
	\label{eq:electroestatica_potencial_integral_contorno}
\end{equation}
donde el segundo término corresponde al potencial en la superficie o el campo eléctrico (es decir la derivada direccional del potencial); este término tiende a cero cuando la  superficie tiende a infinito, recuperando la expresión conocida del potencial. Otro detalle interesante es que si no hay cargas en el volumen, el potencial queda definido por lo que pasa en la superficie \emph{unicamente}.

Para probar la unicidad de esta solución, construimos una nueva solución con dos plausibles, es decir
\[ U = \varphi_1 - \varphi_2 \]
Esta función $U$ nueva verifica que 
\[ \nabla^2 U = 0 \]
en el volumen y el el contorno verifica 
\[ U = 0 \]
o 
\[ \nabla U \cdot \vec{n} = \frac{\partial U}{\partial n} = 0 \]
Si reemplazamos $U$ en la primera identidad de Green tenemos que
\[ \int_{V} (U \nabla^2 U + \nabla U \cdot \nabla U) d\vec{r}' = \oint U \nabla \cdot \vec{n} dS \]
que se reduce a
\[ \int_{V} |\nabla U|^2 d\vec{r}' = 0 \]
con lo que necesariamente implica que
\[ \nabla U = 0\]
es decir que la función $U$ es constante en todo el volumen, y por lo tanto ambas soluciones, $\varphi_1$ y $\varphi_2$ son iguales a menos de una constante (que no tiene significado físico). Esto asegura la \emph{unicidad} de este método.

Otro detalle importante es que no es posible aseverar el valor del potencial y la derivada en el contorno, \emph{solamente} se puede exigir una condición para que la solución sea consistente. Determinar el potencial en la superficie se denomina contorno de \emph{Dirichlet} y aseverar el campo eléctrico, la derivada, se denomina contorno de \emph{Neumann}. Con esto queda claro que la expresión que obtuvimos \emph{no es una solución}, si no una relación integral que debe mantenerse.

Para resolver la ecuación \ref{eq:electroestatica_potencial_integral_contorno} eso, sabemos que la función $\rinv$ verifica 
\begin{equation}
	\nabla'^2 \left(\rinv\right) = -4\pi \delta(\vec{r} - \vec{r}')
\end{equation}
Esta función pertenece a una familia de funciones, llamadas \emph{funciones de Green} $G(\vec{r}, \vec{r}')$ que satisfacen
\begin{equation}
	\nabla'^2 G(\vec{r}, \vec{r}') = - 4\pi \delta(\vec{r} - \vec{r}')
\end{equation}
donde
\begin{equation}
	G(\vec{r}, \vec{r}') = \rinv + F(\vec{r}, \vec{r}')
\end{equation}
con la función $F(\vec{r},\vec{r}')$ verifica
\begin{equation}
	\nabla'^2 F(\vec{r}, \vec{r}') = 0
\end{equation}
Esta función nos permite determinar eliminar alguna de las condiciones, o el potencial o la derivada del potencial, y tener explicitamente las condiciones de contorno. En la segunda identidad de Green, si disponemos $\phi = \varphi$ y $\psi = G(\vec{r}, \vec{r}')$ y usamos la ecuación de Laplace tenemos que
\begin{equation}
	\varphi(\vec{r}) = \int_{V} \rho(\vec{r}') G(\vec{r}, \vec{r}') d\vec{r}' + \frac{1}{4\pi} \oint_{\partial V} \left[ G(\vec{r}, \vec{r}') \frac{\partial \varphi}{\partial n'} - \varphi(\vec{r}) \frac{\partial G}{\partial n'} \right]dS 
\end{equation}
Si elegimos una condición de Dirichlet la función de Green debe ser nula en el contorno, es decir
\begin{equation}
	G_{D}(\vec{r}, \vec{r}') = 0 \qquad \vec{r} \in \partial V
\end{equation}
obteniendo la siguiente expresión integral
\begin{equation}
		\varphi(\vec{r}) = \int_{V} \rho(\vec{r}') G(\vec{r}, \vec{r}') d\vec{r}' - \frac{1}{4\pi} \oint_{\partial V}  \varphi(\vec{r}) \frac{\partial G}{\partial n'} dS
		\label{eq:electroestatica_potencial_green_dirichlet}
\end{equation}
mientras que si tenemos una condicion de Neumann debemos elegir la siguiente condición
\begin{equation}
	\frac{\partial G_{N}(\vec{r},\vec{r}')}{\partial n'} = - \frac{4\pi}{S}
\end{equation}
siendo $S$ la superficie del contorno (esta relación es debido al teorema de la divergencia), y nos queda finalmente
\begin{equation}
		\varphi(\vec{r}) = \langle \varphi \rangle_S + \int_{V} \rho(\vec{r}') G(\vec{r}, \vec{r}') d\vec{r}' + \frac{1}{4\pi} \oint_{\partial V} G(\vec{r}, \vec{r}') \frac{\partial \varphi}{\partial n'} dS 
\end{equation}
donde queda el valor promedio del potencial en la superficie, que no tiene ninguna importancia física.

Hasta ahora definimos a la función de Green a partir de dos parámetros, el punto fuente y el punto campo. Esto es simplemente una cuestión artifiosa, ya que las funciones de Green verifican
\begin{equation}
	G(\vec{r}, \vec{r}') = G(\vec{r}', \vec{r})
\end{equation}
propiedad que se debe verificar siempre, para cualquier frontera o geometría, y que nos permite verificar una expresión obtenida o espolear una solución.

Finalmente, determinar la forma funciona de la función de Green depende de la geometría y existen varios métodos para calcularla (separación de variables e imágenes son los más usados).


\subsection{Medios materiales}

Para medios materiales consideramos medios ponderables, es decir vamos a trabajar con campos promedios, sabiendo que la materia ordinaria tiene una densidad de particulas enorme.
Podemos que si cada elemento de carga está en una situación estática, tenemos
\begin{equation}
\nabla \times \langle \vec{E} \rangle = \nabla \times \vec{E} = 0
\end{equation}
donde eliminamos el promedio para simpleza de la notación. De ahora en adelante serán campos macroscópicos promedio.

Definimos la \emph{polarización eléctrica} como
\begin{equation}
\vec{P}(\vec{r}) = \sum \frac{\vec{N_i}}{V_i} \langle \vec{p}_i \rangle
\end{equation}
donde se promedia en un pequeño volumen centrado en $\vec{r}$, con $N_i$ particulas del tipo $i$ con dipolo eléctrico $\vec{p}_i$, dividido el volumen del cúmulo de partículas. De la misma forma definimos la densidad de carga
\begin{equation}
\rho(\vec{r}) = \sum_i \frac{N_i}{V_i} \langle q_i \rangle + \rho_{libre}
\end{equation}
Se observa que en la materia es neutra, por lo que la carga libre es la que impacta en la densidad de carga.
Si consideramos que no existe otro momento multipolar mayor tenemos que el potencial ponderado, valiendonos del principio de superposición, es
\begin{equation*}
 \varphi(\vec{r}) = \int \rho(\vec{r}') \rinv d\vec{r}' + \int \vec{P}(\vec{r}') \cdot \Rinv d\vec{r}' = \int \rho(\vec{r}') \rinv d\vec{r}' + \int \vec{P}(\vec{r}') \cdot \nabla\left(\rinv\right) d\vec{r}'
\end{equation*}
Si integramos por partes el segundo término, y eliminamos el término de superficie, tenemos que
\begin{equation}
\varphi(\vec{r}) = \int (\rho(\vec{r}') - \nabla' \cdot \vec{P}(\vec{r}')) \rinv d\vec{r'}
\end{equation}
por lo que encontramos una fuente de carga más, que podemos expresar
\begin{equation*}
\nabla \cdot \vec{E} = 4\pi (\rho - \nabla \cdot \vec{P})
\end{equation*}
Acá definimos el campo \emph{desplazamiento eléctrico} como
\begin{equation}
 \vec{D} = \vec{E} + 4\pi \vec{P}
\end{equation}
que nos lleva a la ecuación de Gauss para medios materiales
\begin{equation}
 \nabla \cdot \vec{D} = 4\pi \rho
\end{equation}
Es necesario remarcar que el rotor de $\vec{D}$ no es necesariamente nulo, ya que
\begin{equation}
\nabla \times \vec{D} = \nabla \times \vec{P}
\end{equation}
es decir depende de la polarización del medio.
\end{document}
