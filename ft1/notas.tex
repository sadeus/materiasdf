\documentclass{article}
\usepackage[utf8]{inputenc}
\usepackage[spanish]{babel}
\usepackage{amsmath,amssymb,amsfonts} %Fonts de AMS
\usepackage{mathbbol} %Font para matrices como la identidad
%%Formato de hoja%%
\usepackage[margin=2cm]{geometry}

%%Imagenes%%
\usepackage{graphicx}
\usepackage{wrapfig}
\usepackage{caption}
\usepackage{subcaption}
\graphicspath{{./fig/}}
\usepackage{float}

%%Ecuaciones y teoremas
\numberwithin{equation}{section} %Número de ecuación por sección

%\newtheorem{definition}{Definición}[chapter]
%\newtheorem{axiom}{Axioma}[chapter]
%\newtheorem{law}{Ley}[chapter]
%\newtheorem{principle}{Principio}[chapter]
%\newtheorem{postulate}{Postulado}[chapter]
%\newtheorem{collorary}{Colorario}[chapter]
%\setcounter{\thelaw}{0}
\renewcommand{\vec}[1]{\boldsymbol{#1}}
\newcommand{\bb}[1]{\boldsymbol{#1}}
\newcommand{\rinv}{\frac{1}{|\vec{r} - \vec{r}'|}}
\newcommand{\Rinv}{\frac{\vec{r} - \vec{r}'}{|\vec{r} - \vec{r}'|^3}}

\title{Notas de Física Teórica 1}
\author{S. Schiavinato}
\date{}

\begin{document}
\maketitle
\tableofcontents



\section{Electroestática}

\begin{equation}
\vec{F}_{1\to2} = q_1 q_2\frac{\vec{r}_2 - \vec{r}_1}{|\vec{r}_2 - \vec{r}_1|^3} = - \vec{F}_{2\to1}
\label{eq:electroestatica_coulomb}
\end{equation}

Definimos el campo eléctrico con una carga de prueba $q$ (que no deforma la configuración de cargas presente) como
\begin{equation}
\vec{E} = \frac{\vec{F}}{q}
\label{eq:campo_electrico}
\end{equation}
Experimentalmente vale el principio de superposición, por lo que el campo eléctrico general de una distribución de carga con densidad de carga $\rho(\vec{r}')$ es 
\begin{equation}
\vec{E}(\vec{r}) = \int \rho(\vec{r}') \frac{\vec{r} - \vec{r}'}{|\vec{r} - \vec{r}'|^3} d\vec{r}'
\label{eq:campo_electrico_dist_carga}
\end{equation}
Que se transforma en la expresión para una carga con la siguiente distribución
\begin{equation}
 \rho(\vec{r}) = q \delta(\vec{r} - \vec{r}')
\end{equation}
que es la delta de Dirac. Esta distribución tiene la particularidad de
\begin{equation}
\nabla \cdot \left(\frac{\vec{r} - \vec{r}'}{|\vec{r} - \vec{r}'|^3}\right) = 4\pi\delta(\vec{r} - \vec{r}')
\end{equation}
y además sabemos que
\begin{equation}
\nabla \left(\frac{1}{|\vec{r} - \vec{r}'|}\right) = -\frac{\vec{r} - \vec{r}'}{|\vec{r} - \vec{r}'|^3}
\end{equation}
Con estas dos expresiones podemos calcular la divergencia del campo eléctrico
\[ \nabla \cdot \vec{E}(\vec{r}) = \nabla \cdot \int \rho(\vec{r}') \frac{\vec{r} - \vec{r}'}{|\vec{r} - \vec{r}'|^3} d\vec{r}' = \int \rho(\vec{r}') \, \nabla \cdot \frac{\vec{r} - \vec{r}'}{|\vec{r} - \vec{r}'|^3} d\vec{r}' = 4\pi \int \rho(\vec{r}') \delta(\vec{r} - \vec{r}') d\vec{r}'\]
nos que nos da
\begin{equation}
\nabla \cdot \vec{E}(\vec{r}) = 4\pi \rho(\vec{r})
\label{eq:gauss_diff}
\end{equation}
y el rotor del campo eléctrico es
\[\nabla \times \vec{E}(\vec{r}) = \nabla \times \int \rho(\vec{r}') \frac{\vec{r} - \vec{r}'}{|\vec{r} - \vec{r}'|^3} d\vec{r}' = \int \rho(\vec{r}') \nabla \times \frac{\vec{r} - \vec{r}'}{|\vec{r} - \vec{r}'|^3} d\vec{r}' = - \int \rho(\vec{r}') \nabla \times \nabla \left(\frac{1}{|\vec{r} - \vec{r}'|}\right) d\vec{r}'\]
lo que nos deja 
\begin{equation}
\nabla \times \vec{E}(\vec{r}) = 0
\label{eq:electroestatica_rotor}
\end{equation}
Con esas ecuaciones podemos encontrar el campo eléctrico para cualquier configuración.

Para hacerlo más fácil sabemos que el campo eléctrico debe ser
\begin{equation}
 \vec{E}(\vec{r}) = - \nabla \varphi(\vec{r})
 \label{eq:electroestatico_potencial}
\end{equation}
lo que nos define
\begin{equation}
 \varphi(\vec{r}) = - \int_C \vec{E} \cdot d\vec{l}
 \label{eq:electroestatico_potencial_integral}
\end{equation}
con lo que la divergencia del campo nos da
\begin{equation}
 \nabla^2 \varphi(\vec{r}) = - 4  \pi \rho(\vec{r})
\end{equation}
Que corresponde a la ecuación de Poisson cuando hay carga libre o a la ecuación de Laplace
\begin{equation}
 \nabla^2 \varphi(\vec{r}) = 0
 \end{equation}
 si no hay carga libre. Se presentará algunos métodos de resolución de ambas ecuaciones.
 
\section{Magnetoestática}
Un campo magnético, que puede tener la fuente presente o no en el entorno del problema, produce una fuerza sobre una carga de prueba $q$ igual a 
\begin{equation}
F = q \frac{\vec{v}}{c} \times \vec{B}
\end{equation}
donde el producto $\times$ corresponde al producto vectorial. Esto lo tomamos como la definición del campo magnético, para tener una base mecánica sobre sustentar la teoría.
Experimentalmente, se observa la siguiente ley, Biot-Savart, relacionando el movimiento de cargas o \emph{corrientes} por hilos conductores y el campo magnético generado.
\begin{equation}
 d\vec{B} = \frac{1}{c} \frac{I d\vec{l} \times \vec{r}}{r^3}
\end{equation}
es decir 
\begin{equation}
    \vec{B} = \frac{1}{c} \int  \frac{I d\vec{l} \times \vec{r}}{r^3}.
\end{equation}
Esto se generaliza a corrientes volumétricas de la siguiente forma
\begin{equation}
    \vec{B} = \frac{1}{c} \int  \vec{J}(\vec{r}') \times \Rinv d\vec{r}'
    \label{eq:biot_savart}
\end{equation}

De la misma forma, vamos a buscar expresiones para el rotor y la divergencia de campo magnético. Primero reescribamos un poco la expresión general de la ley de Biot-Savart (eq \ref{eq:biot_savart}), usando la regla del producto del rotor
\[ \nabla \times (\phi \vec{A}) = (\nabla \phi) \times \vec{A} + \phi \nabla \times \vec{A} \]
y que $\vec{J}$ no depende de la coordenada campo $\vec{r}$.
\[ \vec{B} = \frac{1}{c} \int  \vec{J}(\vec{r}') \times \left(- \nabla\left(\rinv\right) \right) d\vec{r}' = \frac{1}{c} \int  \nabla\left(\rinv\right) \times \vec{J}(\vec{r}') d\vec{r}' = \frac{1}{c} \int \nabla \times \left( \frac{\vec{J}(\vec{r}')}{|\vec{r} - \vec{r}'|}\right) d\vec{r}'\]
Como la derivada no depende del integrando, las podemos intercambiar sin problemas, pero a nosotros nos interesa la divergencia de esto
\[ \nabla \cdot \vec{B} = \frac{1}{c} \int  \nabla \cdot \nabla \times \left( \frac{\vec{J}(\vec{r}')}{|\vec{r} - \vec{r}'|}\right) d\vec{r}' = 0\]
es decir 
\begin{equation}
    \nabla \cdot \vec{B} = 0
    \label{eq:gauss_magnetic}
\end{equation}
Para el rotor del campo magnético usamos la siguiente identidad
\begin{equation}
\nabla \times \left( \nabla \times \vec{A} \right) = \nabla(\nabla \cdot \vec{A}) - \nabla^{2}\vec{A}
\end{equation}
que nos queda
\[ \nabla \times \vec{B} = \frac{1}{c} \int  \nabla \times \nabla \times \left( \frac{\vec{J}(\vec{r}')}{|\vec{r} - \vec{r}'|}\right) d\vec{r}' = \frac{1}{c} \int \left\{ \nabla  \left(\vec{J}(\vec{r}') \cdot \nabla \left( \rinv \right) \right) - \vec{J}(\vec{r}') \nabla^2 \left( \rinv \right) \right\} d\vec{r}' \]

Para el segundo término usamos que 
\begin{equation}
    \nabla^2 \phi = \nabla \cdot (\nabla \phi)
\end{equation}
es decir
\begin{equation}
    \nabla^2 \left(\rinv\right) = - 4\pi \delta(\vec{r} - \vec{r}')
\end{equation}
y para el primero integramos por partes
\[ \nabla \times \vec{B} = \frac{1}{c} \nabla \int \frac{\nabla \cdot \vec{J}}{|\vec{r} - \vec{r}'|} d\vec{r}' + \frac{4\pi}{c} \vec{J}(\vec{r})\]
y como estamos en magnetoestática, sabemos que
\begin{equation}
\nabla \cdot \vec{J} = 0
\end{equation}
por lo que nos queda la ley de Ampere, en forma diferencial,
\begin{equation}
\nabla \times \vec{B} = \frac{4\pi}{c} \vec{J}(\vec{r})
\label{eq:ampere}
\end{equation}

Ahora, como el campo magnético $\vec{B}$ es un campo sin divergencia, podemos defir otro campo, llamado \emph{potencial vector magnético} o potencial vector, tal que
\begin{equation}
\vec{B} = \nabla \times \vec{A}
\end{equation}
con el que obtengo la siguiente ecuación diferencial (a partir de la ley de Ampere)
\begin{equation}
\nabla^2 \vec{A} - \nabla (\nabla \cdot \vec{A}) = - \frac{4\pi}{c} \vec{J}
\end{equation}
que se puede simplificar, sabiendo que podemos elegir diferentes potenciales a menos de un gradiente (que se anula al calcularle el rotor)
\begin{equation}
\vec{A} \to \vec{A} + \nabla \varphi
\label{eq:gauge_A}
\end{equation}
con lo que la solución a la ecuación diferencial finalmente nos queda
\begin{equation}
\vec{A} = \frac{1}{c} \int  \frac{\vec{J}(\vec{r}')}{|\vec{r} - \vec{r}'|} d\vec{r}' + \nabla \varphi
\end{equation}
y si elegimos $\varphi$ tal que
\begin{equation}
\nabla \cdot \vec{A} = 0
\label{eq:gauge_coulumb}
\end{equation}
que se denomina \emph{gauge de Coulomb}, tenemos
\begin{equation}
\nabla^2 \vec{A} = - \frac{4\pi}{c} \vec{J}
\end{equation}
que se soluciona con
\begin{equation}
\vec{A} = \frac{1}{c} \int  \frac{\vec{J}(\vec{r}')}{|\vec{r} - \vec{r}'|} d\vec{r}'
\end{equation}
\section{Simetrías}
Con la fuerza de Lorentz
\begin{equation}
\vec{F} = q \left(\vec{E} + \frac{\vec{v}}{c} \times \vec{B}\right)
\end{equation}
aparecen ciertas simetrías, propias de exigir la homogeneidad e isotropía del espacio; es decir, si rotamos o trasladamos las fuentes y las cargas de prueba las fuerzas deben ser iguales, impactando en las fuentes.
Transladar las fuentes implica
\begin{equation}
\rho'(\vec{r}) = \rho(\vec{r} - \vec{a}) \quad \vec{j}'(\vec{r}) = \vec{j}(\vec{r} - \vec{a})
\end{equation}
e implica que las fuerzas
\[q \left(\vec{E}'(\vec{r}) + \frac{\vec{v}}{c} \times \vec{B}'(\vec{r})\right) = q \left(\vec{E}(\vec{r} - \vec{a}) + \frac{\vec{v}}{c} \times \vec{B}(\vec{r} - \vec{a})(\vec{\right)\]
por lo que nos queda (considerando que las cargas de prueba no tienen diferente velocidad)
\begin{equation}
\vec{E}'(\vec{r}) = \vec{E}(\vec{r} - \vec{a}) \quad \vec{B}'(\vec{r}) = \vec{B}(\vec{r} - \vec{a})
\end{equation}

Mientras, rotar las fuentes implica 
\begin{equation}
\rho'(\vec{r}) = \rho(\vec{R}^{-1} \vec{r}) \qquad \vec{j}'(\vec{r}) = \vec{R}\vec{j}(\vec{R}^{-1} \vec{r})
\end{equation}
donde la matriz u operador $\vec{R}$ es una matriz ortogonal que representa la translación. 

Con esta expresión, es evidente que el campo eléctrico se translada como
\begin{equation}
\vec{E}'(\vec{r}) = \vec{R} \vec{E}(\vec{R}^{-1} \vec{r})
\end{equation}
mientras que para el campo magnético debemos usar la siguiente identidad
\begin{equation}
\vec{R} (\vec{A} \times \vec{B}) = (\vec{R} \vec{A]) \times (\vec{R} \vec{B})
\end{equation}
que nos termina dando que el campo magnético rota con las fuentes, salvo un factor, es decir
\begin{equation}
\vec{B}'(\vec{r}) = \det(\vec{R}) \vec{B}(\vec{R}^{-1} \vec{r})
\end{equation}
El factor $\det(\vec{R})$ proviene de considerar las reflexiones, que tienen determinante negativo, y por lo tanto cambian el signo del campo magnético; el campo magnético es un \textbf{pseudo vector}.

Estas simetrías corresponden a las atadas a la realidad mecánica, pero no nos dice que las ecuaciones que relacionan las fuentes con los campos (las ecuaciones de la electrodinámica o de Maxwell) verifican estas simetrías. 

Con cálculo vectorial se demuestra que aún en el caso dinámico las simetrías anteriores se conservan.


 \subsection{Medios materiales}
Para medios materiales consideramos medios ponderables, es decir vamos a trabajar con campos promedios, sabiendo que la materia ordinaria tiene una densidad de particulas enorme.
Podemos que si cada elemento de carga está en una situación estática, tenemos
\begin{equation}
\nabla \times \langle \vec{E} \rangle = \nabla \times \vec{E} = 0
\end{equation}
donde eliminamos el promedio para simpleza de la notación. De ahora en adelante serán campos macroscópicos promedio.

Definimos la \emph{polarización eléctrica} como
\begin{equation}
\vec{P}(\vec{r}) = \sum \frac{\vec{N_i}}{V_i} \langle \vec{p}_i \rangle
\end{equation}
donde se promedia en un pequeño volumen centrado en $\vec{r}$, con $N_i$ particulas del tipo $i$ con dipolo eléctrico $\vec{p}_i$, dividido el volumen del cúmulo de partículas. De la misma forma definimos la densidad de carga
\begin{equation}
\rho(\vec{r}) = \sum_i \frac{N_i}{V_i} \langle q_i \rangle + \rho_{libre}
\end{equation}
Se observa que en la materia es neutra, por lo que la carga libre es la que impacta en la densidad de carga. 
Si consideramos que no existe otro momento multipolar mayor tenemos que el potencial ponderado, valiendonos del principio de superposición, es
\begin{equation*}
 \varphi(\vec{r}) = \int \rho(\vec{r}') \rinv d\vec{r}' + \int \vec{P}(\vec{r}') \cdot \Rinv d\vec{r}' = \int \rho(\vec{r}') \rinv d\vec{r}' + \int \vec{P}(\vec{r}') \cdot \nabla\left(\rinv\right) d\vec{r}' 
\end{equation*}
Si integramos por partes el segundo término, y eliminamos el término de superficie, tenemos que
\begin{equation}
\varphi(\vec{r}) = \int (\rho(\vec{r}') - \nabla' \cdot \vec{P}(\vec{r}')) \rinv d\vec{r'}
\end{equation}
por lo que encontramos una fuente de carga más, que podemos expresar
\begin{equation*}
\nabla \cdot \vec{E} = 4\pi (\rho - \nabla \cdot \vec{P})
\end{equation*}
Acá definimos el campo \emph{desplazamiento eléctrico} como 
\begin{equation}
 \vec{D} = \vec{E} + 4\pi \vec{P} 
\end{equation}
que nos lleva a la ecuación de Gauss para medios materiales
\begin{equation}
 \nabla \cdot \vec{D} = 4\pi \rho
\end{equation}
Es necesario remarcar que el rotor de $\vec{D}$ no es necesariamente nulo, ya que
\begin{equation}
\nabla \times \vec{D} = \nabla \times \vec{P}
\end{equation}
es decir depende de la polarización del medio.
\end{document}

