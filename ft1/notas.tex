\documentclass{article}
\usepackage[utf8]{inputenc}
\usepackage[spanish]{babel}
\usepackage{amsmath,amssymb,amsfonts} %Fonts de AMS
\usepackage{mathbbol} %Font para matrices como la identidad
%%Formato de hoja%%
\usepackage[margin=2cm]{geometry}

%%Imagenes%%
\usepackage{graphicx}
\usepackage{wrapfig}
\usepackage{caption}
\usepackage{subcaption}
\graphicspath{{./fig/}}
\usepackage{float}

%%Ecuaciones y teoremas
\numberwithin{equation}{section} %Número de ecuación por sección

%\newtheorem{definition}{Definición}[chapter]
%\newtheorem{axiom}{Axioma}[chapter]
%\newtheorem{law}{Ley}[chapter]
%\newtheorem{principle}{Principio}[chapter]
%\newtheorem{postulate}{Postulado}[chapter]
%\newtheorem{collorary}{Colorario}[chapter]
%\setcounter{\thelaw}{0}
\renewcommand{\vec}[1]{\boldsymbol{#1}}
\newcommand{\bb}[1]{\boldsymbol{#1}}
\newcommand{\rinv}{\frac{1}{|\vec{r} - \vec{r}'|}}
\newcommand{\Rinv}{\frac{\vec{r} - \vec{r}'}{|\vec{r} - \vec{r}'|^3}}

\title{Notas de Física Teórica 1}
\author{S. Schiavinato}
\date{}

\begin{document}
\maketitle
\tableofcontents

\section{Electroestática}

\begin{equation}
\vec{F}_{1\to2} = k_E \; q_1 q_2\frac{\vec{r}_2 - \vec{r}_1}{|\vec{r}_2 - \vec{r}_1|^3} = - \vec{F}_{2\to1}
\label{eq:electroestatica_coulomb}
\end{equation}

Definimos el campo eléctrico con una carga de prueba $q$ (que no deforma la configuración de cargas presente) como
\begin{equation}
\vec{E} = \frac{\vec{F}}{q}
\label{eq:campo_electrico}
\end{equation}
Experimentalmente vale el principio de superposición, por lo que el campo eléctrico general de una distribución de carga con densidad de carga $\rho(\vec{r}')$ es 
\begin{equation}
\vec{E}(\vec{r}) = k_E \int \rho(\vec{r}') \frac{\vec{r} - \vec{r}'}{|\vec{r} - \vec{r}'|^3} d\vec{r}'
\label{eq:campo_electrico_dist_carga}
\end{equation}
Que se transforma en la expresión para una carga con la siguiente distribución
\begin{equation}
 \rho(\vec{r}) = q \delta(\vec{r} - \vec{r}')
\end{equation}
que es la delta de Dirac. Esta distribución tiene la particularidad de
\begin{equation}
\nabla \cdot \left(\frac{\vec{r} - \vec{r}'}{|\vec{r} - \vec{r}'|^3}\right) = 4\pi\delta(\vec{r} - \vec{r}')
\end{equation}
y además sabemos que
\begin{equation}
\nabla \left(\frac{1}{|\vec{r} - \vec{r}'|}\right) = -\frac{\vec{r} - \vec{r}'}{|\vec{r} - \vec{r}'|^3}
\end{equation}
Con estas dos expresiones podemos calcular la divergencia del campo eléctrico
\[ \nabla \cdot \vec{E}(\vec{r}) = k_E \, \nabla \cdot \int \rho(\vec{r}') \frac{\vec{r} - \vec{r}'}{|\vec{r} - \vec{r}'|^3} d\vec{r}' = k_E \int \rho(\vec{r}') \, \nabla \cdot \frac{\vec{r} - \vec{r}'}{|\vec{r} - \vec{r}'|^3} d\vec{r}' = 4\pi k_E \int \rho(\vec{r}') \delta(\vec{r} - \vec{r}') d\vec{r}'\]
nos que nos da
\begin{equation}
\nabla \cdot \vec{E}(\vec{r}) = 4\pi k_E \rho(\vec{r})
\label{eq:gauss_diff}
\end{equation}
y el rotor del campo eléctrico es
\[\nabla \times \vec{E}(\vec{r}) = k_E \nabla \times \int \rho(\vec{r}') \frac{\vec{r} - \vec{r}'}{|\vec{r} - \vec{r}'|^3} d\vec{r}' = k_E \int \rho(\vec{r}') \nabla \times \frac{\vec{r} - \vec{r}'}{|\vec{r} - \vec{r}'|^3} d\vec{r}' = -k_E \int \rho(\vec{r}') \nabla \times \nabla \left(\frac{1}{|\vec{r} - \vec{r}'|}\right) d\vec{r}'\]
lo que nos deja 
\begin{equation}
\nabla \times \vec{E}(\vec{r}) = 0
\label{eq:electroestatica_rotor}
\end{equation}
Con esas ecuaciones podemos encontrar el campo eléctrico para cualquier configuración.

Para hacerlo más fácil sabemos que el campo eléctrico debe ser
\begin{equation}
 \vec{E}(\vec{r}) = - \nabla \varphi(\vec{r})
 \label{eq:electroestatico_potencial}
\end{equation}
lo que nos define
\begin{equation}
 \varphi(\vec{r}) = - \int_C \vec{E} \cdot d\vec{l}
 \label{eq:electroestatico_potencial_integral}
\end{equation}
con lo que la divergencia del campo nos da
\begin{equation}
 \nabla^2 \varphi(\vec{r}) = - 4  \pi k_E \rho(\vec{r})
\end{equation}
\end{document}


