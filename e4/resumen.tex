\documentclass[a4paper]{article}

\usepackage[spanish]{babel}
\usepackage{amsmath}
\usepackage[utf8]{inputenc}
\usepackage{braket}

\usepackage{slashed}
\usepackage{lineno}

\newcommand{\Lagr}{\mathcal{L}}
\newcommand{\du}{\partial}

\title{Resumen E4}
\author{ Sebastián Schiavinato}
\date{\today}

\begin{document}

\maketitle

\tableofcontents

\section{Introducción}\label{ sec:introduccion}

Cómo vamos a trabajar con mecánica relativista y mecánica cuántica, durante este resumen vamos a tomar


\noindent\noindent\[
\begin{aligned}
c &= 1\\
\hbar &= 1
\end{aligned}
\]


\section{Teoría nuclear}\label{sec:nuclear}
La física nuclear empieza con el descubrimiento de la radioactividad por Becquerel en 1896. 

Rutherford con esta radiación la orientó sobre materia y en 1911 propuso el primer modelo atómico. Propuso un nucleo cargado positivamente, con los electrones, descubiertos por Thompson en 1904, orbitando al núcleo. 

Luego Chadwick descubrirá el neutrón, que permitie explicar la masa del nucleo y la existencia de isótopos, es decir átomos con misma cantidad de electrones (que determinan la química del compuesto) pero diferente masa.


\section{Isospin y Quarks}\label{sec:isospin-y-quarks}
Con los avances experimentales se observaron muchas partículas con diferente carga y masa

Observando la relación de dispersión de varias partícuas 
\[ \ket{\psi} = \ket{\psi}_S^{SU(3)} \otimes \ket{\chi}_S^{SU(2)} \otimes \ket{\phi}_A^{SU(3)} \]

\section{Revisión de Relatividad especial}\label{sec:relatividad-especial}

\begin{equation}
E^2 = m^2 + p^2
\label{eq:energy_momentum_relativistic}
\end{equation}


\section{Ecuación de Klein Gordon}\label{sec:klein-gordon}
La cuantización canónica define a la energía (Hamiltoneano) y al momento como

\[
\begin{aligned}
E &= H = i \hbar \du_t \\
\bar{p} &= i \nabla
\end{aligned}
\]

por lo que la ecuación
\[H = \frac{p^2}{2m}\]
pasa a ser

\begin{equation}
H = i \du_t = - \frac{\nabla^2}{2m}
\label{eq_schrodinger}
\end{equation}

que puede ser aplicada a una función  $\psi$  cualquiera.

 Ahora la relación energía-momento relativista (ecuación~\ref{eq:energy_momentum_relativistic})
tendría la siguiente cuantización canónica
\[
- \du^2_{tt} = - \nabla^2 + m^2
\]
que finalmente la podemos expresar covariantemente
\[
\du_\mu \du^\mu + m^2 = 0
\]
que se llama ecuación de Klein Gordon

 Esta ecuación tiene la problemática que definir una corriente de probablidad no definida positiva, que proviene de segunda derivada temporal (ya que la transforma en una ecuación hiperbólica)

\section{Ecuación de Dirac}\label{sec:dirac}


\noindent Dirac propuso un hamiltoneano lineal en el momento, es decir
\[H = p_i \alpha_i + \beta m\]
con  $\alpha$  vector y  $\beta$ número en principio a determinar


\section{Teoría de campos clásica}\label{ sec:campos}
\begin{equation}
\du_\mu \left(\du_{\du_\mu\phi} \Lagr\right) - \du_{\phi} \Lagr = 0
\label{eq:euler_lagrange}
\end{equation}


\subsection{Teorema de Noether}\label{ sec:noether}
Si la transformación infinitecimal (que depende de un parámetro  $\epsilon$ )  $\psi \to \tilde{\psi}$  deja invariante al lagrangiano a menos de una divergencia, es decir
\[\Lagr = \tilde{\Lagr} + \du_\mu H^\mu,\]
entonces existe una corriente 
\[J^\mu = \delta \phi \du_{\du_\mu \phi} \Lagr - H^\mu,\]
tal que 
\[\du_\mu J^\mu = 0.\]


\subsection{Lagrangianos}\label{ sec:lagrangianos}


\noindent Fermi Dirac
\begin{equation}
\Lagr = i \bar{\psi} \gamma^{\mu} \du_\mu \psi - m \bar{\psi} \psi
\label{eq:fermionic}
\end{equation}

\noindent Klein Gordon
\begin{equation}
\Lagr = \du_\mu \du^\mu \phi - m^2 \phi^* \phi
\end{equation}
Si usamos la ecuación~\ref{eq:euler_lagrange}


\subsection{Simetrías globales}\label{sec:simetras-globales}


\subsection{Simetrías locales}\label{sec:simetras-locales}

\begin{equation}
\du_\mu \to D_\mu = \du_\mu - i q A_\mu
\end{equation}


\section{Electrodinámica}\label{sec:electrodinamica}


\section{Fuerza fuerte}\label{sec:fuerza-fuerte}


\section{Fuerza débil}\label{sec:fuerza-debil}


\section{Mecanismo de Higgs}\label{sec:mecanismo-de-higgs}


\section{Modelo estándar}\label{sec:modelo-estandar}
\[
 \Lagr = - \frac{1}{4} F_{\mu \nu} F^{\mu \nu} + i \bar{\psi} \slashed{D} \psi - \|D_\mu \phi\|^2 - V(\phi) 
\]

\nocite{*}
\bibliographystyle{unsrt}
\bibliography{resumen}

\end{document}
