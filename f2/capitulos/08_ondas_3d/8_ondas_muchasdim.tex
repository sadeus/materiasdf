\documentclass[a4paper,spanish]{article}


\usepackage[spanish]{babel}
\usepackage[latin1]{inputenc}
\usepackage{amsmath}
\usepackage{amssymb}
\usepackage[margin=1.5cm]{geometry}
\usepackage{graphicx}
\usepackage{caption}
\usepackage{subcaption}
\usepackage{float}
\newcommand{\oiint}{\displaystyle\bigcirc\!\!\!\!\!\!\!\!\int\!\!\!\!\!\int}


\usepackage{epsfig}
\usepackage{color}
\usepackage{amsfonts}
\usepackage[T1]{fontenc}

\def\Fou {\mathcal{F}}
\def\Rea {\mathcal{R}e}
\def\Ima {\mathcal{I}m}
\def\N {\mathbb{N}}
\def\C {\mathbb{C}}
\def\Q {\mathbb{Q}}
\def\R {\mathbb{R}}
\def\Z {\mathbb{Z}}


%\renewcommand{\contentsname}{\'Indice}
%\renewcommand{\chaptername}{Cap\'\i tulo}
%\renewcommand{\bibname}{Referencias}

\newtheorem{prop}{Proposici\'on}[section]
\newtheorem{teo}[prop]{Teorema}
\newtheorem{defi}[prop]{Definici\'on}
\newtheorem{obs}[prop]{Observaci\'on}
\newtheorem{cor}[prop]{Corolario}
\newtheorem{lema}[prop]{Lema}
\newtheorem{ejem}[prop]{Ejemplo}
\newtheorem{ejer}[prop]{Ejercicio}

\numberwithin{equation}{section}
\newtheorem{definition}{Definici\'on}


\newenvironment{proof}{
\trivlist \item[\hskip \labelsep\mbox{\it Demostraci\'on:
}]}{\hfill\mbox{$\square$}
%\trivlist \item[\hskip \labelsep{\sl
%#1}\mbox{Demostraci\'on}]}{\hfill\mbox{$\square$}
\endtrivlist}

%\topmargin 0cm \oddsidemargin 0.7cm %% margenes
%\textheight 21cm \textwidth 15cm %% tama\~no del texto
\parindent 0cm %% sangria

\begin{document}

\part{Ondas en varias dimensiones}
	Para generalizar escribimos, de forma natural, la ecuaci\'on de ondas cl\'asica en el espacio euclidio.
	\begin{equation}
		\nabla^2 \psi(\textbf{r},t) = \frac{1}{c^2} \partial_{tt} \psi(\textbf{r},t)
		\label{eq:ondas_ecuacion_general}
	\end{equation}
	donde el operador $\nabla^2$ se denomina laplaciano y vale
	\begin{equation}
		\nabla^2 = \partial_{xx} + \partial_{yy} + \partial_{zz}
		\label{eq:laplaciano_cartesianas}
	\end{equation}
	La soluci\'on que vamos a manejar corresponde a la propagaci\'on de una onda en un sentido (fuera de la fuente) con perfil arm\'onico, es decir
	\begin{equation}
		\psi(\textbf{r},t) = A(\textbf{r}) e^{i (\textbf{k} \cdot \textbf{r} - \omega t)}
		\label{eq:ondas_ecuacion_general_solucion}
	\end{equation}
	donde el numero de onda $k$ pasa a ser un vector que determina la direcci\'on de propagaci\'on de la onda, cuyo modulo sigue verifcando la relaci\'on de dispersi\'on. De esa forma podemos tener varios frentes de ondas posibles, siendo un frente de onda la regi\'on del espacio con la misma fase a un mismo tiempo (es decir la regi\'on del espacio que verifica $\textbf{k} \cdot \textbf{r} = \text{cte}$). 
	
	Si tenemos una fuente puntual el frente de onda ser\'a esf\'erico podemos ensayar en la ecuaci\'on de ondas la siguiente soluci\'on, que es isotr\'opica, \[\psi(\textbf{r},t) = f(r) e^{i (\omega t - k r)}\] con lo que obtenemos \[\frac{df}{dr} + \frac{f}{r} = 0\] que tiene por soluci\'on
	\begin{equation}
		A(\textbf{r}) = \frac{A}{r}
		\label{eq:ondas_ecuacion_general_amplitud_esferica}
	\end{equation}
	es decir que la amplitud decae con la distancia como una homogr\'afica. La intensidad de la onda la podemos encontrar como el cuadrado de la amplitud, por lo que la intensidad de cae como $r^{-2}$, como es de esperar ya que la superficie de una esfera de propoprcional al radio cuadrado.
	
	Podemos hacer el mismo an\'alisis para una fuente cilindrica, pero es m\'as complejo, por lo que presentamos la soluci\'on final
	\begin{equation}
		A(\textbf{r}) = \frac{A}{\sqrt{\rho}}.
		\label{eq:ondas_ecuacion_general_amplitud_cilindrica}
	\end{equation}
	Finalmente un fente de onda plano tiene la siguiente amplitud
	\begin{equation}
		A(\textbf{r}) = A
		\label{eq:ondas_ecuacion_general_amplitud_plana}
	\end{equation}
	es decir un fente de onda plano mantiene la amplitud
	\subsection{Reflexi\'on de ondas planas}
		Consideremos que una onda plana incide en una superficie plana interfaz de dos medios con diferentes velocidades de propagaci\'on (en este punto la onda plana ayuda al an\'alisis simplificado pero las ecuaciones tienen validez general). El $\textbf{k}$ est\'a contenido en el plano xz y la interfaz est\'a dispuesta en el eje $z = 0$, lo que nos permite resolver el problema ayudandonos de la simetr\'ia.
		
		La onda incidente tiene la siguiente forma (considerando que no se desplaza en el eje $y$) \[ \psi(x,y,z,t) = A e^{i (\omega t - k_x x - k_z z)}\] y en el punto $z = 0$ valdr\'a \[\psi(x,y,z=0,t) = A e^{i(\omega t - k_x x)}\] como esperabamos. Sabemos que $k^2 = k_x^2 + k_y^2 + k_z^2$, por lo que $k_z = \pm \sqrt{k^2 - k_x^2 - k_y^2}$, es decir existe dos posibles rayos que verifican la relaci\'on en $z = 0$, una es la incidente y otra es la reflejada. La onda reflejada se propaga con el mismo $k_x$ y $k_z$ con signo cambiado, por lo que tiene el mismo \'angulo respecto a la normal al plano, es decir
		\begin{equation}
			\theta_i = \theta_r
			\label{eq:ondas_reflexion_ley}
		\end{equation}
		ecuaci\'on que se conoce con la ley de reflexi\'on especular.
		
		Consideremos que la onda transmitida sufre un cambio en su amplitud dependiendo de un coeficiente de transmisi\'on $T$ independiente de la posici\'on o del tiempo, el cual va a depende finalmente de la f\'isica de la interfaz y la onda. Adem\'as que por continuidad el n\'umero de ondas en el eje x debe ser igual de ambos lados. Es decir que la se\~nal transmitida ser\'a \[ \psi_t(x,y,0,t) = A T e^{i (\omega t - k_{i_x} x)} = B e^{i(\omega t - k_{2_x})}\]
		lo que nos lleva a que 
		\begin{equation}
			k_{1_x} = k_{2_x} = k_1 \sen(\theta_i) = k_2 \sen(\theta_t) = n_1 \sen(\theta_i) = n_2 \sen(\theta_t)
			\label{eq:ondas_transmision_snell}
		\end{equation}
		donde definimos $n_i = \frac{c k_i}{\omega} = \frac{c_i}{c}$ como el indice de refracci\'on.
		
		\end{document}