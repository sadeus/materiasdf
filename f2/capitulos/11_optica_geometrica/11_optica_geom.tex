\documentclass[a4paper,spanish]{article}


\usepackage[spanish]{babel}
\usepackage[latin1]{inputenc}
\usepackage{amsmath}
\usepackage{amssymb}
\usepackage[margin=1.5cm]{geometry}
\usepackage{graphicx}
\usepackage{caption}
\usepackage{subcaption}
\usepackage{float}
\newcommand{\oiint}{\displaystyle\bigcirc\!\!\!\!\!\!\!\!\int\!\!\!\!\!\int}


\usepackage{epsfig}
\usepackage{color}
\usepackage{amsfonts}
\usepackage[T1]{fontenc}

\def\Fou {\mathcal{F}}
\def\Rea {\mathcal{R}e}
\def\Ima {\mathcal{I}m}
\def\N {\mathbb{N}}
\def\C {\mathbb{C}}
\def\Q {\mathbb{Q}}
\def\R {\mathbb{R}}
\def\Z {\mathbb{Z}}


%\renewcommand{\contentsname}{\'Indice}
%\renewcommand{\chaptername}{Cap\'\i tulo}
%\renewcommand{\bibname}{Referencias}

\newtheorem{prop}{Proposici\'on}[section]
\newtheorem{teo}[prop]{Teorema}
\newtheorem{defi}[prop]{Definici\'on}
\newtheorem{obs}[prop]{Observaci\'on}
\newtheorem{cor}[prop]{Corolario}
\newtheorem{lema}[prop]{Lema}
\newtheorem{ejem}[prop]{Ejemplo}
\newtheorem{ejer}[prop]{Ejercicio}

\numberwithin{equation}{section}
\newtheorem{definition}{Definici\'on}


\newenvironment{proof}{
\trivlist \item[\hskip \labelsep\mbox{\it Demostraci\'on:
}]}{\hfill\mbox{$\square$}
%\trivlist \item[\hskip \labelsep{\sl
%#1}\mbox{Demostraci\'on}]}{\hfill\mbox{$\square$}
\endtrivlist}

%\topmargin 0cm \oddsidemargin 0.7cm %% margenes
%\textheight 21cm \textwidth 15cm %% tama\~no del texto
\parindent 0cm %% sangria

\begin{document}

\part{\'Optica geom\'etrica}
		Esta \'optica es estudio de elementos \'opticos solo con consideraciones geometricas (ya que podemos despreciar la longitud de onda $\lambda$ de la onda). En esta secci\'on vamos a analizar como alterar frentes de ondas por medio de elementos refractantes y reflectantes, despreciando efectos f\'isicos de la onda (ver \ref{sec:difraccion}).
		
		Para este estudio riguroso vamos a usar dos principios de gran importancia no solamente en la \'optica sino tambi\'en en otras ramas. Primeramente, tenemos el principio de Huygens (que despu\'es vamos a completar y formalizar en la secci\'on \ref{sec:difraccion}) que determina que cada frente de onda lo podemos describir como una sucesi\'on de fuentes esf\'ericas (con las mismas frecuencias que el frente) que generan la envolvente del frente; este principio tiene algunas limitaciones, pero se puede probar que es una consecuencias de la ecuaci\'on de onda. El segundo principio que vamos a utilizar es un principio variacional (el primero, hist\'oricamente, utilizado), el principio de Fermat, que pregona que el camino \'optico que finalmente recorre un haz en ir de un punto $A$ a un punto $B$ es la soluci\'on estacionaria respecto a variaciones de dicho y en general consideraremos el camino \'optico m\'inimo (ya que no existe m\'aximo). Este principio lo podemos escribir as\'i
		\begin{equation}
			\delta S = \delta \int_{A}^{B} n(s) ds = 0
			\label{eq:optica_fermat}
		\end{equation}
		es decir que la variaci\'on de esa integral sea nula, por lo que el camino \'optico (que podemos notar $S$ o $LCO$, dependiendo del contexto) es estacionario. El principio se puede utilizar para encontrar la ley de Snell, considerando que le tiempo que tarda de ir de un punto a otro es m\'inimo, es decir \[ t = \frac{\vec{AO}}{v_1} + \frac{\vec{OB}}{v_2} = \frac{\sqrt{h^2+x^2}}{v_1} + \frac{\sqrt{b^2+(a - x)^2}}{v_2}\] siendo $h$ la altura desde la interfaz al punto $A$, $x$ la distancia paralela a la interface del punto $A$ al punto de contacto, $b$ la altura desde la interfaz hasta el punto $B$ y $a$ la distancia paralela a la interfaz del punto $A$ al punto $B$. El \'unico parametro es la distancia $x$, por lo que debemos derivar el tiempo respecto a dicho \[\frac{d t}{d x} = \frac{x}{v_1 \sqrt{h^2 + x^2}} + \frac{ -(a - x)}{v_2 \sqrt{b^2 + (a - x)^2}} = 0\] lo que finalmente queda (considerando que $n = \frac{c}{v_1}$).
		\begin{equation}
			n_1 \sen(\theta_1) = n_2 \sen(\theta_2)
			\label{eq:snell}
		\end{equation}
		la ley de Snell
	
	\subsection{Lentes}
		Una lente es un sistema \'optico refractor, es decir un sistema con una discontinuidad en el indice de refracci\'on respecto al medio, que varia la distribuci\'on de la energ\'ia incidente. Con esta definici\'on podemos clasificar como lentes variados objetos en variados espectros de rayos (UV, IR, microondas y hasta sonido). En este curso vamos a analizar lentes esf\'ericas, sin considerar lentes asfericas (que se pueden analizar por medio del principio de Fermat y la geometr\'ia anal\'itica) ya que son de muy dificiles de producir.
		
		Analizemos un rayo que incide en una superf\'icie esf\'erica de radio $R$, el camino \'optico que efect\'ua es \[LCO = n_1 l_o + n_2 l_i\] que podemos escribir utilizando el teorema del coseno de la siguiente manera \[LCO = n_1\sqrt{R^2 + (s_o + R)^2 - 2 R (s_o + R) \cos(\phi)} + n_2\sqrt{R^2 + (s_i - R)^2 - 2 R (s_i - R) \cos(\phi)},\] siendo $s_o$ la distancia entre el vertice (el punto m\'as cercano al objeto, por donde pasa el eje \'optico de la superficie) y la fuente, llamado distancia objeto, mientras que $s_i$ es la distancia imagen, donde se une el eje \'optico y el haz refractado. El \'angulo $\phi$ es el \'angulo del radio vector del punto de incidencia del haz, que va a ser el parametro del camino \'optico, es decir que vamos a hacer $\frac{d LCO}{d\phi} = 0$ lo que nos queda 
		\[\frac{n_1 R (s_o + R) \sen(\phi)}{2 l_0} - \frac{n_2 R (s_i - R) \sen(\phi)}{2 l_i} = 0 \qquad \Rightarrow \qquad \frac{n_1}{l_o} + \frac{n_2}{l_i} = \frac{1}{R}(\frac{n_2 s_i}{l_i} - \frac{n_1 s_o}{l_o}.\] 
		Esta expresi\'on es muy complicada de llevar a la pr\'actica, por lo que se hace una aproximaci\'on a primer orden, que corresponde con $\sen(\phi) \approx \phi$ y los rayos son paraxiales, es decir $l_x = s_x$ para imagen y objeto, lo que se llama \'optica gaussiana. De esta forma nos queda
		\begin{equation}
			\frac{n_1}{s_o} + \frac{n_2}{s_i} = \frac{n_2 - n_2}{R}
			\label{eq:lentes_esferica}
		\end{equation}
		Podemos definir dos distancias importantes, denominadas foco objeto y foco imagen, que corresponden al punto donde el frente de onda converge en o del infinito respectivamente (es decir que el foco objeto corresponde a la distancia objeto donde el frente de onda en la lente es plano y el foco imagen corresponde a la distancia imagen que se forma al tener una un frente de onda plano incidente). Por lo tanto siguien la siguiente relaci\'on 
		\begin{equation}
			 f_o = \frac{n_1}{n_2 - n_1} R \qquad f_i = \frac{n_2}{n_2 - n_1} R
			 \label{eq:lentes_foco_objeto_imagen}
		\end{equation}
		
		Otra clasificaci\'on importante es objeto e imagen virtual o real. Un objeto es real cuando los rayos convergen del objeto, mientras que es virtual si los rayos convergen hacia el. Lo contrario pasa para la imagen. En nuestra convenci\'on, la m\'as usual en \'optica geom\'etrica, las distancias objeto e imagen son positivas si son reales (lo mismo aplica para el foco) y negativa si son virtuales.
		
		Las lentes que vamos a analizar ahora son de dobre interfaz, en general dos indices de refracci\'on, el del medio y el de la lente. La imagen de la primera interfaz va a pasar ser el objeto de la segunda interfaz, por lo que podemos considerar que $|s_{o_2}| = |s_{i_2}| + d$, siendo $d$ el espesor de la lente entre vertices, y considerando la convenci\'on propuesta obtenemos $s_{o_2} = - s_{i_1} + d$ por lo que al sumar la relaci\'on \ref{eq:lentes_esferica} para ambas interfaces obtenemos
		\begin{equation}
			n_{e} \left(\frac{1}{s_{o_1}} + \frac{1}{s_{i_2}}\right) = (n_l - n_e)\left(\frac{1}{R_1} - \frac{1}{R_2}\right) + \frac{n_l d}{(s_{i_1} - d) s_{i_1}}
			\label{eq:lentes_general}
		\end{equation}
		y si despreciamos el valor $d$, es decir consideramos lentes delgadas, nos queda (considerando que las distancias son las mismas medidas desde cualquier v\'ertice)
		\begin{equation}
			\frac{1}{s_o} + \frac{1}{s_i} = \frac{(n_l - n_e)}{n_e} \left(\frac{1}{R_1} + \frac{1}{R_2}\right)
			\label{eq:lentes_constructores}
		\end{equation}
		y podemos deducir que el foco imagen y foco objeto son iguales, por lo que obtenemos que
		\begin{equation}
			\frac{1}{s_o} + \frac{1}{s_i} = \frac{1}{f}
			\label{eq:lentes_guassiana}
		\end{equation}
		denominada f\'ormula gaussiana para lentes delgadas.
		
		Consideremos un objeto de extensi\'on finita, que por abstraccion lo vemos como un conjunto infinito de puntos. De un gr\'afico podemos deducir, considerando que la altura debajo del eje \'optico es negativa, que
		\begin{equation}
			M_T = \frac{y_o}{y_i} = -\frac{s_i}{s_o}
			\label{eq:lentes_aumento_transversal}
		\end{equation}
		que es el aumento transversal de la imagen. Con un poco de trabajo algebraico podemos llegar a la formula newtoneana
		\begin{equation}
			x_o x_i = f^2
			\label{eq:lentes_newton}
		\end{equation}
		donde $x$ es la distancia del objeto o imagen al foco (donde queda evidente, considerando la convenci\'on, que como el $f^2$ es siempre positivo, la imagen y el objeto deben estar opuestos respecto al respectivo punto focal). El aumento lateral o longitudinal de una lente se define y vale lo siguiente
		\begin{equation}
			M_L = \frac{d x_i}{d x_o} = - \frac{f^2}{x_0^2} = - M_T
			\label{eq:lentes_aumento_lateral}
		\end{equation}
		es decir que si la imagen est\'a derecha los objetos parecen m\'as cerca de lo que realmente est\'an, que es lo que se logra con una lupa por ejemplo.
		
		\subsection{Espejos}
		Un espejo es un elemento puramente reflector, para el cual vamos a estudiar \'unicamente con la ley de reflexi\'on \ref{eq:ondas_reflexion_ley}. La aproximaci\'on que vamos a tomar es que no absorben nada de la radiaci\'on incidente, es decir son espejos perfectos. 
		
		Analizar un espejo plano es una tarea muy sencilla, ya que lo \'unico que hace es reflejar perfectamente sin alterar la imagen m\'as que por un giro o inversi\'on de los sentidos (observable en un simple trazado de rayos). De esta forma vamos a analizar espejos esf\'ericos (dejando los asfericos de lado, pero dichos son estudiables f\'acilmente con la ley de reflexi\'on y geometr\'ia anal\'itica). Veamos un espejo concavo, que por convenci\'on tiene radio negativo (ya que el centro de la superficie se haya a la izquierda del vertice). La imagen puntual (ya que el objeto utilizado es puntual, ubicado en $s_o$) que se va a generar es de origen real, por lo que ser\'a positiva (esta es la conveci\'on utlizada tambi\'en para lentes), cruzar\'a el eje optico a una distancia $s_i$ del vertice. Considerando rayos paraxiales, es decir que la distancia entre el punto de contacto y la distancia objeto son aproximadamente iguales y lo mismo para la distancia imagen, podemos probar, por argumentos gr\'aficos, que la siguiente expresi\'on tiene validez
		\begin{equation}
			\frac{1}{s_o} + \frac{1}{s_i} = - \frac{2}{R}
			\label{eq:espejos_esfericos}
		\end{equation}
		y de donde deducimos que el foco es igual a
		\begin{equation}
			f_o = f_i = - \frac{R}{2}
			\label{eq:espejos_esfericos_foco}
		\end{equation}
		por lo que podemos deducir que
		\begin{equation}
		 \frac{1}{s_o} + \frac{1}{s_i} = \frac{1}{f}
		 \label{eq:espejos_esfericos_ecuacion}
		\end{equation}
		donde queda evidente que podemos utilizar todas las ecuaciones derivadas para las lentes delgadas por similitud matem\'atica, considerando la convenci\'on de signos correspondiente.
		
		%FALTA PRISMAS

\end{document}