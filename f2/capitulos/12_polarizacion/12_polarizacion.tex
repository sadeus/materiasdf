\documentclass[a4paper,spanish]{article}


\usepackage[spanish]{babel}
\usepackage[latin1]{inputenc}
\usepackage{amsmath}
\usepackage{amssymb}
\usepackage[margin=1.5cm]{geometry}
\usepackage{graphicx}
\usepackage{caption}
\usepackage{subcaption}
\usepackage{float}
\newcommand{\oiint}{\displaystyle\bigcirc\!\!\!\!\!\!\!\!\int\!\!\!\!\!\int}


\usepackage{epsfig}
\usepackage{color}
\usepackage{amsfonts}
\usepackage[T1]{fontenc}

\def\Fou {\mathcal{F}}
\def\Rea {\mathcal{R}e}
\def\Ima {\mathcal{I}m}
\def\N {\mathbb{N}}
\def\C {\mathbb{C}}
\def\Q {\mathbb{Q}}
\def\R {\mathbb{R}}
\def\Z {\mathbb{Z}}


%\renewcommand{\contentsname}{\'Indice}
%\renewcommand{\chaptername}{Cap\'\i tulo}
%\renewcommand{\bibname}{Referencias}

\newtheorem{prop}{Proposici\'on}[section]
\newtheorem{teo}[prop]{Teorema}
\newtheorem{defi}[prop]{Definici\'on}
\newtheorem{obs}[prop]{Observaci\'on}
\newtheorem{cor}[prop]{Corolario}
\newtheorem{lema}[prop]{Lema}
\newtheorem{ejem}[prop]{Ejemplo}
\newtheorem{ejer}[prop]{Ejercicio}

\numberwithin{equation}{section}
\newtheorem{definition}{Definici\'on}


\newenvironment{proof}{
\trivlist \item[\hskip \labelsep\mbox{\it Demostraci\'on:
}]}{\hfill\mbox{$\square$}
%\trivlist \item[\hskip \labelsep{\sl
%#1}\mbox{Demostraci\'on}]}{\hfill\mbox{$\square$}
\endtrivlist}

%\topmargin 0cm \oddsidemargin 0.7cm %% margenes
%\textheight 21cm \textwidth 15cm %% tama\~no del texto
\parindent 0cm %% sangria

\begin{document}

\part{Polarizaci\'on}
	De las ecuaciones de Maxwell obtenemos que la onda electromagn\'etica, la luz, es una onda vectorial transversal, es decir que es una onda con diferentes planos de oscilaciones posibles. En esta secci\'on analizaremos que sucede cuando dos ondas linealmente polarizadas con fases diferentes son colineales, y por lo tanto se suman. Esto genera un cambio en el estado de polarizaci\'on y veremos como se puede describir. 

	Para simplificar consideremos los campos el\'ectricos (ya que el campo magn\'etico es perpendicular en los casos que vamos a tratar) est\'an contenidos en un plano (que denominamos xy), es decir que en general \[ \textbf{E} = (\hat{x} E_x + \hat{y} E_y) e^{i(k z - \omega t)}\] donde $E_x$ e $E_y$ son magnitudes complejas que permite agregarle una fase a al campo en una coordenada. Si los $E_x$ e $E_y$ son reales, la polarizaci\'on es claramente lineal, girada dependiendo del valor de cada campo. Mientras si $\displaystyle E_y = \|E_x\| e^{i \left(-\frac{pi}{2}\right)}$, obtenemos polarizaci\'on circular (ya que $E_x^2 + E_y^2 = 1$) en este caso derecha (se vueve siguiendo las agujas de reloj respecto a la fuente) y si la fase es $\frac{pi}{2}$ obtenemos una polarizaci\'on circular izquierda (para analizar esto r\'apidamente se debe ver como cambia el campo con el tiempo).
	
	El caso general es la polarizaci\'on eliptica, que corresponde a una diferencia de fase en alguna componente y magnitud de componentes diferentes, es decir 
	\begin{align*}
		E_x &= \|E_x\| e^{i(k z - \omega t)}\\
		E_y &= \|E_y\| e^{i (k z - \omega t + \epsilon)}
	\end{align*}
	Podemos escribir la siguiente expresi\'on, que se deduce facilmente (ac\'a tomamos la parte real de la onda y utilizamos la identidad trigonom\'etrica de la suma de \'angulos) \[\frac{E_y}{\|E_y\|} - \frac{E_x}{\|E_x\|} \cos(\epsilon) = - \sen(k z - \omega t) \sen(\epsilon) = - \sqrt{1 - \frac{E_x}{\|E_x\|}} \sen(\epsilon)\]
	donde al final se hizo uso de la ecuaci\'on del campo en $x$ m\'as la identidad trigonom\'etrica de Pit\'agoras. Finalmente nos queda la siguiente expresi\'on
	\begin{equation}
		\left(\frac{E_y}{\|E_y\|}\right)^2 + \left(\frac{E_x}{\|E_x\|}\right)^2 - 2 \left(\frac{E_y}{\|E_y\|}\right) \left(\frac{E_x}{\|E_x\|}\right) \cos(\epsilon) = \sen^2(\epsilon)
		\label{eq:polarizacion_elipitica}
	\end{equation}
	que tiene validez general (ya que para la polarizaci\'on circular $\epsilon = \frac{pi}{2}$ y queda una circunferencia, mientras si es lineal $\epsilon = 0$ quedando una relaci\'on lineal). De la ecuaci\'on anterior obtenemos el \'angulo del eje $x$ de la elipse respecto a la base 
	\begin{equation}
		tg(2\alpha) = \frac{2 \|E_x\| \|E_y\| \cos(\epsilon)}{\|E_x\|^2 - \|E_y\|^2}
		\label{eq:polarizacion_eliptica_angulo}
	\end{equation}
	
	La polarizaci\'on de una fuente natural es un tema especial, ya que las fuente generan trenes de ondas con cierta longitud (que se denomina de coherencia) con una polarizaci\'on definida por un tiempo determinado (tiempo de coherencia), por lo que la luz natural no tiene una polarizaci\'on definida en el tiempo (denominada polarizaci\'on al azar). Mientras un tren de ondas infinito, o una onda monocrom\'atica, debe estar siempre polarizado. La realidad f\'isica corresponde a ambas situaciones, lo que se denomina parcialmente polarizado.
	
	Ahora pasamos a describir los diferentes elementos capaces de cambiar la polarizaci\'on de la luz incidente, denominados polarizadores. Analizemos primeramente un polarizador lineal, del que podemos deducir ,por medio de un esquema simple, que el campo el\'ectrico despu\'es del polarizador es \[E = E_0 \cos(\theta) \hat{n}\] donde $\hat{n}$ es la direcci\'on en el plano xy del eje del polarizador, por lo que la intensidad a la salida ser\'a
	\begin{equation}
		I(\theta) = I(0) \cos^2(\theta)
		\label{eq:polarizacion_malus}
	\end{equation}
	que se denomina la ley de Malus. 
	
	Veamos algunos tipos de polarizaci\'on por medio de elementos materiales, donde tenemos elementos dicroicos y bifringentes (lo que son polarizadores por transmisi\'on). Un elemento dicroico es un material que absorbe una componente del campo el\'ectrico, por medio de una fuerte anisotrop\'ia de su estructura, y deja pasar la luz en un eje denominado \'optico; ejemplos de elementos dicroicos son los cristales de turmalina, o las l\'aminas polarizadoras lineales (llamadas polaroides). Un material bifringente exhibe anistrop\'ia en el indice de refracci\'on, en general en dos ejes (uno \'optico u ordinario y uno extraordinario), que permite que se generen dos imagenes; el funcionamiento de un elemento bifringente es id\'entico a un material dicroico, separando de la luz que sale una componente del campo el\'ectrico.
	
	Luego tenemos la polarizaci\'on por reflexi\'on en medios diel\'etricos, para el cual tenemos que hacer un an\'alisis de la din\'amica del problema. Al incidir una onda electromagnetica con una polarizaci\'on definida logramos que los dipolos moleculares se acomplen con dicha perturbaci\'on, obteniendo una onda reflejada. Si la onda es incide justamente con el campo el\'ectrico perpendicular al los dipolos no se reflejar\'a onda alguna y todo ser\'a transmitido. Ese \'angulo se denomina \'angulo de Brewster y es f\'acilmente deducible sabiendo que $\theta_t + \theta_B = 90^\circ$ (ya que la onda transmitida va ser ortogonal a los dipolos del material)
	\begin{equation}
		\theta_B = \arctg\left(\frac{n_2}{n_1}\right)
		\label{eq:polarizacion_brewster}
	\end{equation}
	
	Finalmente nos queda un elemento \'optico importante cuando se describe la polarizaci\'on, los retardadores, que b\'asicamente agregan una fase determinada a uno de los ejes (el eje r\'apido), cambiando la polarizaci\'on de la onda incidente. Existen retardadores de media onda, onda completa y cuarto de onda que agregan una fase de, respectivamente, $2\pi$, $\pi$ y $\frac{\pi}{2}$ a la componente del eje r\'apido, pudiendo obtener a la salida una polarizaci\'on eliptica, circular o lineal dependiedo de cada caso.
	

\end{document}