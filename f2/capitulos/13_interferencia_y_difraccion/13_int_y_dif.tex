\documentclass[a4paper,spanish]{article}


\usepackage[spanish]{babel}
\usepackage[latin1]{inputenc}
\usepackage{amsmath}
\usepackage{amssymb}
\usepackage[margin=1.5cm]{geometry}
\usepackage{graphicx}
\usepackage{caption}
\usepackage{subcaption}
\usepackage{float}
\newcommand{\oiint}{\displaystyle\bigcirc\!\!\!\!\!\!\!\!\int\!\!\!\!\!\int}


\usepackage{epsfig}
\usepackage{color}
\usepackage{amsfonts}
\usepackage[T1]{fontenc}

\def\Fou {\mathcal{F}}
\def\Rea {\mathcal{R}e}
\def\Ima {\mathcal{I}m}
\def\N {\mathbb{N}}
\def\C {\mathbb{C}}
\def\Q {\mathbb{Q}}
\def\R {\mathbb{R}}
\def\Z {\mathbb{Z}}


%\renewcommand{\contentsname}{\'Indice}
%\renewcommand{\chaptername}{Cap\'\i tulo}
%\renewcommand{\bibname}{Referencias}

\newtheorem{prop}{Proposici\'on}[section]
\newtheorem{teo}[prop]{Teorema}
\newtheorem{defi}[prop]{Definici\'on}
\newtheorem{obs}[prop]{Observaci\'on}
\newtheorem{cor}[prop]{Corolario}
\newtheorem{lema}[prop]{Lema}
\newtheorem{ejem}[prop]{Ejemplo}
\newtheorem{ejer}[prop]{Ejercicio}

\numberwithin{equation}{section}
\newtheorem{definition}{Definici\'on}


\newenvironment{proof}{
\trivlist \item[\hskip \labelsep\mbox{\it Demostraci\'on:
}]}{\hfill\mbox{$\square$}
%\trivlist \item[\hskip \labelsep{\sl
%#1}\mbox{Demostraci\'on}]}{\hfill\mbox{$\square$}
\endtrivlist}

%\topmargin 0cm \oddsidemargin 0.7cm %% margenes
%\textheight 21cm \textwidth 15cm %% tama\~no del texto
\parindent 0cm %% sangria

\begin{document}


\part{Interferencia y Difracci\'on}

\section{Interferencia}
	\label{sec:interferencia}
	El proceso de interferencia consiste en la superposici\'on de una o m\'as ondas, y vamos a hacer hincapie en las ondas electromagn\'eticas (teniendo que considerar la naturaleza vectorial de las ondas). Para empezar el an\'alisis consideremos dos fuentes puntuales monocrom\'aticas separadas m\'as de una longitud de onda (para despreciar efectos de difracci\'on) y veamos como interaccionan en un punto lo suficientemente alejado (por distancia real o por medio de lentes) para que sea un frente plano. En ese punto el campo el\'ectrico ser\'a \[\textbf{E} = \textbf{E}_{1} + \textbf{E}_2 = \textbf{E}_{o_1} e^{i (\textbf{k}_1 \cdot \textbf{r} - \omega t + \epsilon_1)} + \textbf{E}_{o_2} e^{i (\textbf{k}_2 \cdot \textbf{r} - \omega t + \epsilon_2)}.\] Aca consideramos que las ondas tiene el mismo plano de polarizaci\'on (y son linealmente polarizadas) para simplificar el an\'alisis. Consideremos que la intensidad del campo el\'ectrico como el cuadrado de dicha magnitud vectorial, es decir \[E^2 = \textbf{E} \cdot \textbf{E} = \|\textbf{E}_1\|^2 + \|\textbf{E}_2\|^2 + 2 \textbf{E}_1 \cdot \textbf{E}_2 = I_1 + I_2 + 2 \textbf{E}_{o_1} \cdot \textbf{E}_{o_2} e^{i \omega t} e^{ i (\epsilon_1 + \epsilon_2)} e^{i (\textbf{k}_1 + \textbf{k}_2) \cdot \textbf{r}},\] y si consideramos el promedio temporal de la intensidad (lo que se llama irradiancia relativa) obtenemos que el termino final no depende del tiempo (ya que integramos en un periodo mucho m\'as grande que $\dfrac{2\pi}{\omega}$) y que la irradiancia queda 
	\begin{equation}
		I = \langle\|\textbf{E}_1\|^2\rangle_T + \langle\|\textbf{E}_1\|^2\rangle_T + \textbf{E}_{o_1} \cdot \textbf{E}_{o_2} \cos((\textbf{k}_1 - \textbf{k}_2)\cdot \textbf{r} + \epsilon_1 - \epsilon_2)
		\label{eq:interferencia_general}
	\end{equation}
	y en el caso que los campos tengan el mismo plano de oscilaci\'on nos queda que
	\begin{equation}
		I = I_1 + I_2 + 2 \sqrt{I_1 I_2} \cos(\delta)
		\label{eq:interferencia_particular}
	\end{equation}
	donde $\delta$ representa la diferencia de camino \'optico y de fase inicial de las ondas que se superponen. Un caso de mucha importancia practica es cuando las fuentes tienen la misma irradiancia, lo que terminamos obteniendo la siguiente expresi\'on
	\begin{equation}
		I = 2 I_0 \cos^2\left(\frac{\delta}{2}\right)
		\label{eq:interferencia_patron_clasico}
	\end{equation}
	Los puntos donde la interferencia es m\'axima o minimia (llamada, respectivamente, destructiva o constructiva) forman una superficie que podemos encontrar de la siguiente forma (para un frente de onda esf\'erico producto de una fuente puntual, considerando que la distancia es suficientemente grande para despreciar cambios de la amplitud)
	\begin{align}
		\delta = 2n\pi \qquad &\Rightarrow \qquad r_1 - r_2 = \frac{2n\pi + \epsilon}{k} = n\lambda + \frac{\epsilon}{k}\\
		\delta = (2n + 1)\pi \qquad &\Rightarrow \qquad r_1 - r_2 = \frac{(2n + 1)\pi + \epsilon}{k}
	\end{align}
	ecuaciones que corresponden hiperboloides de revoluci\'on.
	
	El an\'alisis que hicimos ac\'a corresponde a dos ondas de extensi\'on infinita, la cual es generada por una fuente de forma continua. Pero en general las fuentes generan de forma intermitente, durante un tiempo determinado, que vamos a llamar tiempo de coherencia. La coherencia es la capacidad de interferir de forma correcta con dos fuentes, que se va a dar dependiendo solamente de las frecuencias temporales y la frecuencias espaciales (adem\'as de la polarizaci\'on). Volviendo, el tiempo de coherencia va a determinar las componentes del paquete de onda, por lo que va a determinar qu\'e tan pura va a ser el espectro y por lo tanto que tan visibles van a ser las franjas (ya que solo interfieren ondas de la misma frecuencia temporal, como ya vimos). La coherencia espacial, mientras tanto, determina cuanto varia en el tiempo la fase relativa entre dos puntos del espacio fijos, que va a venir determinado por la forma del frente de onda y la fuente. Tambi\'en la coherencia se da, para una onda vectorial como la luz, si la superposici\'on comparten componentes de polarizaci\'on, como vimos en la ecuaci\'on \ref{eq:interferencia_general}
	
	Existen varios tipos de interfer\'ometros, que dividimos por el tipo de proceso f\'isico que efectuan, divisi\'on de frente de onda o divisi\'on de amplitud. 
	
	El intefer\'ometro por divisi\'on de frente de onda m\'as com\'un es el de Young, el cual consiste en una fuente coherente que pasa por dos ranuras muy estrechas. Haciendo el an\'alisis geom\'etrico, considerando la aproximaci\'on paraxial de los rayos obtenemos que la franjas est\'an distanciadas
	\begin{equation}
		\Delta y = \frac{L}{d}\lambda
		\label{eq:interferencia_young_interfranja}
	\end{equation}
	que obtenemos sabiendo que la diferencia de camino \'optico tiene que ser igual a $m\lambda$, lo que obtenemos que los m\'aximos est\'an en 
	\begin{equation}
		y_m = \frac{L}{d} m \lambda
		\label{eq:interferencia_young_maximos}
	\end{equation}
	y por lo tanto la irradiancia ser\'a
	\begin{equation}
		I = 4 I_o \cos^2\left(\frac{y d \pi}{L\lambda}\right)
		\label{eq:interferencia_young_irradiancia}
	\end{equation}
	
	Los dem\'as interfer\'ometros por divisi\'on de frente de onda se puede reducir al interfer\'ometro de Young, haciendo consideranciones respecto a las fases de las fuentes (ya que si se usa espejos la fase cambia en $\pi$ si es razante, es decir \'angulo de incidencia mayor a 90$^\circ$). Podemos enumerar en estos interfer\'ometros al espejo de Lloyd (donde se genera una fuente nueva con un espejo donde incide razante la fuente), el doble espejo de Fresnel (que son dos espejos planos que generan dos fuentes a partir de una, que terminar interfiriendo en una zona definida del espacio) o el biprisma de Fresnel (un primsa doble que genera una franja.
	
	Los interfer\'ometros por divisi\'on de amplitud consisten en dividir la onda en dos partes por medio de un espejo, y luego unirlas (generando una diferencia de camino \'optico) en alg\'un punto. Vamos a diferenciar interfer\'ometros por reflexi\'on pura o con refracci\'on (estos \'ultimos son creados por pel\'iculas diel\'ectricas). 
	
	Analizaremos primeros los de refracci\'on, que corresponde a la interferencia producto de una pel\'icula diel\'ectrica. Parte de los haces (consideramos que son originarios de una fuente puntual coherente) son reflejadados adem\'as de transmitidos, por lo que analizando un haz incidente observamos un reflejado y luego un transmitdo que tambi\'en ser\'a reflejado y que finalmente sald\'a con el mismo \'angulo que incidi\'o; si logramos con una lente converger el primer rayo y el segundo rayo reflejado tendremos franjas de interferencia dependientes de la diferencia de camino \'optico, que finalmente depende del \'angulo de incidencia (por lo que se llaman franjas de igual inclinaci\'on) de la siguiente manera
	\begin{equation}
		\Delta LCO = 2 n d \cos(\theta_t)
		\label{eq:interferencia_franjas_inclinacion}
	\end{equation}
	De esta forma es posible utilizar una fuente extensa para lograr franjas con la misma inclinaci\'on, siempre que se logre la coherencia temporal.
	
	El otro tipo de franjas se denomina de igual espesor y se logra considerando el camino \'optico de un haz que indice sobre una pel\'icula de diferentes espesores, como puede ser una mancha de aceite sobre el agua. Debido a que el camino \'optico varia dependiendo del espesor de pel\'icula, var\'ia el patr\'on. Interfer\'ometros de este estilo pueden ser el de cu\~na, que genera un patr\'on sobre una una de las caras, o los anillos de Newton.
	
	Un interfer\'ometro puramente por reflexi\'on, es decir con espejos solamente, es una variaci\'on del interfer\'ometro de Michelson, el cual usando dos espejos y un divisor de haz (que podemos pensar como un espejo semiplateado, genera dos fuentes virtuales que podemos ir corriendo obteniendo diferentes patrones de interferencia.
	
	Otros tipos de interfer\'ometros generan interferencia por medio de varios haces, como ser el interfer\'ometro de Fabry-Perot.
\section{Difracci\'on}
	\label{sec:difraccion}
	Difracci\'on es un proceso donde la luz (o cualquier onda) se desv\'ian de su recorrido recto, debido a que se encuentra con una obst\'aculo fijo. Tenemos que considerar que el principio de Huygens no considera esta situaci\'on, lo que pasa al obstruir el paso de una onda, por lo que tenemos que completarlo, como hizo Fresnel. El principio de Huygens-Fresnel determina que todo punto de un frente de onda sin obstrucci\'on inmediata es una fuente de ondas esf\'ericas de la misma frecuencia que la onda inicial y la amplitud de la onda un tiempo posterior va a ser la superposici\'on de los trenes de ondas nuevos (a\'un considerando amplitud y fase).
	
	Vamos a despreciar los aspectos electromagn\'eticos de la difracci\'on con una pantalla opaca (que denominamos $\Sigma$) con una abertura, donde la pantalla generar\'a ondas reflejadas por la oscilaci\'on de sus dipolos. Si la longitud de onda es relativamente comparable con la abertura veremos en la cercan\'ia (lo que se denomina campo cercano o difracci\'on de Fresnel) una sombra parecida a la fuente, con algunas deformaciones. Al ir alejando la pantalla encontramos que en un momento la sombra genera un patr\'on extendido que no var\'ia m\'as (solamente de amplitud). Este caso se denomina campo lejano o difracci\'on de Fraunhoffer. Lo que pasa en la difracci\'on de Fresnel es que la superposici\'on de ondas, que depende de la fase, tiene una componente angular, alterando de forma considerablemente el campo, pero a larga distancia solo depende de la posici\'on donde se forma la franja y la forma de las franjas depender\'an solamente de la forma de la abertura de forma lineal.
	
	Para estudiar el proceso de difracci\'on vamos a considerar una l\'inea de $N$ osciladores todos coherentes distanciados $d$. Consideramos que el campo es lejano, por lo que usamos todos haces paralelos. La suma ser\'a
	\[E = E_o(r) \sum_{j = 1}^{N} e^{i(k r_j - \omega t)} = E_o(r) e^{i(k r_1 - \omega t} \left(1 + \sum_{j = 2}^N e^{i k (r_j - r_1)}\right) = E_o(r) e^{i(k r_1 - \omega t} \left(1 + \sum_{j = 2}^N \left(e^{i \delta}\right)^2\right)\] donde usamos que $\delta = k d \sen(\theta)$, por lo que es deducible que $n \delta = k(r_n - r_1)$. La serie anterior converge en la siguiente expresi\'on \[E = E_o(r)e^{-i \omega t} e^{i \left(k r_1 + (N -1)\frac{\delta}{2}\right)}\frac{\sen\left(\frac{N \delta}{2}\right)}{\sen\left(\frac{\delta}{2}\right)}\] y por lo tanto la intensidad ser\'a \[I = I_o \frac{\sen^2(N\delta/2)}{\sen^2(\delta/2)}\] donde observamos que para $N = 2$ obtenemos el patr\'on de interferencia (ecuaci\'on \ref{eq:interferencia_patron_clasico})
	Ahora consideremos una fuente lineal, es decir el caso anterior pero la distancia entre fuentes es diferencial. Un diferencial de campo lo podemos escribir \[dE = \frac{\varepsilon}{R} e^{i(k r - \omega t)} dy\] donde definimos la eficacia de la fuente $\varepsilon = r \|E\|$ y consideramos que el vector $r \approx R$, el del centro de la fuente al punto en cuesti\'on. Para la fase de la fuente debemos escribir la distancia \[r = \sqrt{R^2 + y^2 - 2 R y \sen(\theta)} = R - y\sen(\theta) + \frac{y^2}{2R} \cos^2(\theta) + \dots\] donde usamos los primeros terminos de la expansi\'on. Lo que nos queda finalmente es la siguiente integral \[ E(P) = \frac{\varepsilon}{R} \int_{-b/2}^{b/2} e^{i(k (R - y\sen(\theta)) - \omega t)} dy\] que finalmente vale (considerar que $\sen(\theta)$ es constante)
	\begin{equation}
		E = \frac{\varepsilon b}{R} \frac{\sen(k b / 2 \sen(\theta))}{k b / 2 \sen(\theta)} e^{i (k R - \omega t)}
		\label{eq:difraccion_campo_ranura}
	\end{equation}
	y por lo tanto la irradiancia de una dicha fuente ser\'a
	\begin{equation}
		I(k b / 2 \sen(\theta) = \beta) = I(0) \text{senc}^2(\beta)
		\label{eq:difraccion_ranura}
	\end{equation}
	El an\'alisis ac\'a llevado a cabo es v\'alido para una rendija fina, despreciando la difracci\'on debido al tama\~no de dicha. 
	
	Para $N$ cantidad de rendijas podemos encontrar que el campo distante es igual al encontrado para una rendija, multiplicado por la expresi\'on que obtuvimos para los $N$ osciladores, es decir
	\begin{equation}
		E = C \text{senc}(\beta) e^{i(-k R + \omega t - (N - 1)\alpha)} \frac{\sen(N \alpha)}{\sen(\alpha)}
		\label{eq:difraccion_red_rendijas}
	\end{equation}
	donde $\alpha = \frac{k a \sen(\theta)}{2}$, siendo $a$ la distancia entre rendijas. De esta forma el patr\'on que se obtiene es el producto del patr\'on de una rendija con el patr\'on generado por una fuente puntual en cada rendija, resoluci\'on que tiene de caracter general al considerar la \'optica de Fourier. La irradiancia resultante ser\'a por lo tanto
	\begin{equation}
		I(\theta) = I(0) \text{senc}^2(\beta) \left(\frac{\sen(N\alpha)}{\sen(\alpha)}\right)^2
		\label{eq:difraccion_red_rendijas_irradiancia}
	\end{equation}
	expresi\'on funcional que observamos en la figura \ref{fig:difraccion_red_rendijas_irradiancia}, donde vemos que los m\'aximos principales se encuentran donde la interferencia es m\'axima ($\alpha = k \pi$), mientras entre m\'aximos habr\'a $N - 1$ m\'aximos secundarios ($\alpha = \frac{(2 n + 1)\pi}{2N}$ que maximiza el denominador de la parte de interferencia). Finalmente observamos que la interferencia es modulada por la figura de difracci\'on.
	
	
	Para resolver aberturas bidimensionales consideremos un diferencial de campo producto de un diferencial de superficie \[ dE = \frac{\varepsilon}{R} e^{i (k r - \omega t)} dS.\] En este caso la distancia $r$ seguir\'a $r^2 = X^2 + (Y - y)^2 + (Z - z)^2$, donde las coordenadas en may\'uscula representan el punto $P$ y las minusculas las coordenadas del plano de la abertura; de esta forma $r^2 = R^2 + (x^2 +z^2) - 2(Yy - Zz)$ y si hacemos un desarrollo de Taylor a primer orden obtenemos que \[r = R \left(1 - \frac{Yy - Zz}{R^2}\right)\] por lo que la integral nos queda
	\begin{equation}
		E = \frac{\varepsilon}{R} e^{i(\omega t - k R)} \int_{\text{abertura}} e^{i k \frac{Yy - Zz}{R}} dS
		\label{eq:difraccion_abertura_bidimensional}
	\end{equation}
	que podemos escribir de la siguiente manera, consolidando las constantes en la funci\'on $A(x,y)$ y observamos que $k_Y = k \frac{Y}{R}$ y lo mismo para $k_Z$
	\begin{equation}
		E(x,y) = \iint A(x',y') e^{i (k_Y y + k_Z z)} dx dy
		\label{eq:difraccion_general}
	\end{equation}
	donde queda evidente que el campo finalmente va a ser la transformada de Fourier de la funci\'on $A$ que llamamos funci\'on de abertura.
	
	Haciendo la integral \ref{eq:difrracion_abertura_bidimensional} para una ranura rectangular, con lado $a$ en el eje $z$ y lado b en el eje $y$, encontramos que tiene una irrandiancia compuesta por dos irradiancias en el eje $y$ y en el eje $z$ (ya que la integral se puede separar en virtud del teorema de Fubini), es decir
	\begin{equation}
		I(\beta_y = k b Y / 2 R, \beta_y = k a Z / 2 R) = I(0) \text{senc}^2(\beta_y) \text{senc}^2(\beta_z)
	\end{equation}
	Para una abertura circular, radio $a$, debemos escribir la integral en coordenadas esf\'ericas y utilizar la funci\'on $J_0$ de Bessel, que es igual a \[J_0(u) = \frac{1}{2\pi} \int_0^{2\pi} e^{i u \cos(v)} dv,\] la cual al integrarla nos queda un campo igual a \[ E = \frac{\varepsilon e^{i(\omega t - k R)}}{k q} 2 \pi a J_1\left(\frac{k a q}{R}\right)\] donde $q$ es la distancia en el plano imagen al punto $P$. De esta forma la irradiancia queda 
	\begin{equation}
		I(\theta) = 4 I_0 \frac{J_1^2(k a \sen(\theta))}{k a \sen(\theta)}
		\label{eq:difraccion_circular_irradiancia}
	\end{equation}
	Ac\'a podemos observar que el m\'aximo central tiene una extensi\'on radial, que num\'ericamente observamos igual a 
	\begin{equation}
		q_1 = 1,22\frac{R\lambda}{2 a}
	\end{equation}
	valor que va a determinar, por ejemplo, el l\'imite de una lente que enfoca en la pantalla (es decir $f \approx R$) debido a la difracci\'on. 
	La expresi\'on anterior  determina el criterio de Rayleigh para resolver imagenes; en este criterio es necesario que el m\'aximo central del patr\'on de cada imagen est\'e el en minimo de la otra, as\'i es posible diferenciar las dos fuentes.
	\subsection{Redes de difracci\'on}
		Una red de difracci\'on corresponde a un objeto con un patr\'on periodico que difracta de forma periodica la luz entrante, por medio de cambio periodo de fase u/o amplitud. Para las redes de amplitud, utilizamos la irradiancia de la ecuaci\'on \ref{eq:difraccion_red_rendijas_irradiancia}, ya que una colecci\'on de $N$ rendijas es b\'asicamente una red de difracci\'on por amplitud (sea por reflexi\'on o transmisi\'on). De esa forma nos queda que
		\begin{equation}
			a (\sen(\theta_m) - \sen(\theta_i)) = m \lambda
			\label{eq:difraccion_red_ecuacion}
		\end{equation}
		que es de caracter general, a\'un para las redes por cambio de fase (donde podemos encontrar la diferencia de camino \'optico por medio de un esquema), donde $\theta_i$ es el \'angulo de incidencia respecto a la normal de la red.
		
		Finalmente definimos la resoluci\'on espectral de una red como la capacidad de generar m\'aximos para longitudes de onda cercanas, es decir
		\begin{equation}
			R = \frac{\lambda}{\Delta \lambda_{\text{min}}}
			\label{eq:difraccion_red_resolucion_espectral}
		\end{equation}
		y si consideramos el criterio de Rayleigh (que el m\'aximo central de una longitud de onda est\'e en el primer minimo de la m\'as cercana) obtenemos que \[R = m N,\] que finalmente nos queda
		\begin{equation}
			R = \frac{N a(\sen(\theta_m) - \sen(\theta_i)}{\lambda}
			\label{eq:difraccion_red_resolucion_rayleigh}
		\end{equation}
		donde podemos deducir que en autocolimaci\'on ($\theta_i = -\theta_m = 90^\circ$) logramos la mayor resoluci\'on espectral.
		
		\end{document}