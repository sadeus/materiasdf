\documentclass[a4paper,spanish]{article}


\usepackage[spanish]{babel}
\usepackage[latin1]{inputenc}
\usepackage{amsmath}
\usepackage{amssymb}
\usepackage[margin=1.5cm]{geometry}
\usepackage{graphicx}
\usepackage{caption}
\usepackage{subcaption}
\usepackage{float}
\newcommand{\oiint}{\displaystyle\bigcirc\!\!\!\!\!\!\!\!\int\!\!\!\!\!\int}


\usepackage{epsfig}
\usepackage{color}
\usepackage{amsfonts}
\usepackage[T1]{fontenc}

\def\Fou {\mathcal{F}}
\def\Rea {\mathcal{R}e}
\def\Ima {\mathcal{I}m}
\def\N {\mathbb{N}}
\def\C {\mathbb{C}}
\def\Q {\mathbb{Q}}
\def\R {\mathbb{R}}
\def\Z {\mathbb{Z}}


%\renewcommand{\contentsname}{\'Indice}
%\renewcommand{\chaptername}{Cap\'\i tulo}
%\renewcommand{\bibname}{Referencias}

\newtheorem{prop}{Proposici\'on}[section]
\newtheorem{teo}[prop]{Teorema}
\newtheorem{defi}[prop]{Definici\'on}
\newtheorem{obs}[prop]{Observaci\'on}
\newtheorem{cor}[prop]{Corolario}
\newtheorem{lema}[prop]{Lema}
\newtheorem{ejem}[prop]{Ejemplo}
\newtheorem{ejer}[prop]{Ejercicio}

\numberwithin{equation}{section}
\newtheorem{definition}{Definici\'on}


\newenvironment{proof}{
\trivlist \item[\hskip \labelsep\mbox{\it Demostraci\'on:
}]}{\hfill\mbox{$\square$}
%\trivlist \item[\hskip \labelsep{\sl
%#1}\mbox{Demostraci\'on}]}{\hfill\mbox{$\square$}
\endtrivlist}

%\topmargin 0cm \oddsidemargin 0.7cm %% margenes
%\textheight 21cm \textwidth 15cm %% tama\~no del texto
\parindent 0cm %% sangria


\begin{document}

\part{Relaciones de dispersi\'on y ondas en medios dispersivos}

\section{Ondas especiales}
	El an\'alisis hecho hasta ahora, en general, asumi\'o perturbaciones que son soluci\'on de la ecuaci\'on de onda cl\'asica. Ahora vamos a analizar un caso particular donde la ecuaci\'on de ondas no es la cl\'asica y adem\'as es dispersiva. 
	
	Consideremos $N$ p\'endulos con misma masa acoplados con resortes de constante $k$. Hacemos el an\'alisis din\'amico (aplicado las leyes de Newton) considerando que el n\'umero $N$ es muy grande (as\'i como se hizo en la secci\'on \ref{sec:oscilador_recurrencia}).
	\[ m \ddot{\psi}_n = - \frac{m g}{l} \psi_n + k (\psi_{n+1} - \psi_n) - k ( \psi_n - \psi_{n+1}) \]
	y si hacemos el pasaje al continuo obtenemos \[\psi_{n+1} \to \psi(x + a, t) = \psi(x, t) + a \partial_x \psi + a^2 \partial_{xx} \psi\] y lo mismo para $\psi_{n-1}$, con lo que obtenemos
	\begin{equation}
		\partial_{tt} \psi(x,t) = - \omega_0^2 \psi(x,t) + \frac{K a^2}{m} \partial_{xx} \psi(x,t)
		\label{eq:ondas_klein_gordon}
	\end{equation}
	que se denomina ecuaci\'on, de ondas, de Klein-Gordon. Si proponemos la soluci\'on $\psi(x,t) = A(x) e^{i (\omega t + \phi)}$ obtenemos
	\[ \frac{k a^2}{m} \frac{d^2 A}{d z} - (\omega_0^2 - \omega^2)A(z) = 0\]
	ecuaci\'on que tiene soluci\'on exponencial pura (soluci\'on que decimos est\'a en la zona reactiva) para $\omega > \omega_0$ y una soluci\'on arm\'onica para $\omega < \omega_0$ (decimos que es la zona activa). 
	
	Para encontrar la relaci\'on de dispersi\'on propongamos una soluci\'on al caso discreto y luego hagamos el paso al l\'imite continuo. Si proponemos $\psi_n = A_n e^{i \omega t}$, obtenemos la relaci\'on \[m \omega^2 A_n = \frac{m g}{l} A_n - K (A_{n+1} + A_{n-1} - 2 A_n),\] para la cual proponemos $A_n = A e^{i k a n}$, con lo que nos queda \[m \omega^2 = \frac{m g}{l} + 4 K \sen^2\left(\frac{k a}{2}\right)\] que en el l\'imite nos queda
	\begin{equation}
		\omega^2 = \frac{g}{l} + \frac{K a^2}{m} k^2 = \omega_0^2 + c^2 k^2
		\label{eq:ondas_klein_gordon_dispersion}
	\end{equation}
	es decir que que existe una dispersi\'on, ya que $k \neq c^{-1} \omega$.
	
\subsection{Paquetes de ondas}
			Veamos como se mueven dos ondas arm\'onicas, propagandose para el mismo sentido, con frecuencias muy parecidas, es decir que la pertubaci\'on ser\'a
			\begin{equation*}
				\psi(x,t) = A e^{i(\omega_1 t - k_1 x)} + B e^{i(\omega_2 t - k_2 x)}
			\end{equation*}
			Para eso definimos la frecuencias promedios y las desviaciones
			\begin{equation}
				\omega = \frac{\omega_1 + \omega_2}{2} \quad \Delta \omega = \frac{\omega_1 - \omega_2}{2} \qquad k = \frac{k_1 + k_2}{2} \quad \Delta k = \frac{k_1 - k_2}{2}
				\label{eq:ondas_paquetes_batidos_frecuencias}
			\end{equation}
			por lo que la pertubaci\'on queda
			\begin{equation}
				\psi(x,t) = (A + B) e^{i (\omega t - k x)} \left(e^{i(\Delta \omega t - \Delta k x)} + e^{-i(\Delta \omega t - \Delta k x)}\right) = (A + B) e^{i(\omega t - k x)} \cos(\Delta \omega t - \Delta k x)
				\label{eq:ondas_paquetes_batidos}
			\end{equation}
			es decir que se da una pulsaci\'on de la ampltidud de la perturbaci\'on con frecuencia $\omega$ y $k$. 
			
			La velocidad de cada una de estas componentes es igual a $c = \frac{\omega}{k}$. En ese caso la velocidad del paquete modulante, que denominamos velocidad de grupo, es
			\begin{equation}
				c_g = \frac{\omega_1 - \omega_2}{k_1 - k_2} = \frac{\Delta \omega}{\Delta k} = \frac{d \omega}{d k}.
				\label{eq:ondas_paquetes_velocidad_grupo}
			\end{equation}
			Si las perturbaciones se mueven a diferentes velocidades entonces (ya que la relaci\'on de dispersi\'on no es lineal) la modulaci\'on se va deformando con el tiempo, adem\'as que la velocidad de grupo puede ser m\'as r\'apida o m\'as lenta que la velocidad de fase. Para ver eso consideremos que la velocidad de grupo es una funci\'on del n\'umero de onda o la frecuencia angular, es decir
			\[ c_g = c_g(\omega) = c_g(k) \quad \Rightarrow \quad \Delta c_g = c_g(k) - c_g(k_0) = \left.\frac{d c_g}{d k}\right|_{k_0} \Delta k = \left.\frac{d^2 \omega}{d k^2}\right|_{k_0}  \Delta k\]
			por lo tanto la diferencia espacial es
			\[ \Delta x = x - x(t=0) = \left.\Delta x\right|_{t = 0} + \Delta c_g t\]
			es decir que si la velocidad de grupo cambia la frecuencia espacial o temporal el paquete se va deformando con el paso del tiempo.
			
			La ecuaci\'on de onda que permite que suceda este evento se denomina dispersiva, ya que el paquete de onda se va deformando en el tiempo.
			
			Ahora pasamos a describir una perfil de onda general, o paquete de onda, por medio del an\'alisis de Fourier. Un perfil de onda f\'isicamente posible (que sea continuo y que tenga longitud mucho menor que el medio de propagaci\'on) se puede descomponer en una base del espacio (en este caso vamos a utilizar exponenciales $e^{i \omega t}$ o $e^{i k x}$)
			\begin{equation}
				\psi(x,t) = \frac{1}{2\pi} \int_{-\infty}^{\infty}\int_{-\infty}^{\infty} A(\omega) B(k) e^{i(k x - \omega t)} d\omega dk = \mathcal{F}^{-1}[\hat{\psi}(k,t)] \mathcal{F}^-1[\hat{\psi}(x,\omega)]
				\label{eq:ondas_paquetes_fourier_general}
			\end{equation}
			donde definimos la transformada de Fourier $\mathcal{F}$ y su inversa $\mathcal{F}^{-1}$ de la siguiente forma
			\begin{equation}
				f(t) = \frac{1}{\sqrt{2 \pi}} \int_{-\infty}^{\infty} \hat{f}(\omega) e^{i \omega t} d\omega \qquad \hat{f}(\omega) = \frac{1}{\sqrt{2 \pi}} \int_{-\infty}^{\infty} f(t) e^{i \omega t} dt
				\label{eq:ondas_paquetes_fourier_def}
			\end{equation}
			donde llamamos espectro a la funci\'on $\hat{f}$. 
			
			Podemos usar tablas de transformadas para encontrarlas, ya que una la transformada es lineal, y tambi\'en verifica algunas propiedades m\'as interesantes (ver ap\'endice). Una propiedad muy importante es la relaci\'on entre el ancho de banda en frecuencia y en tiempo (o en frecuencia espacial y la posici\'on), que determina una relaci\'on de incerteza; una se\~nal muy definida en frecuencia no va a estar definida en tiempo, como por ejemplo una delta de Dirac 
			\[\delta(x) = \begin{cases} \infty & x = 0 \\ 0 & x \neq 0 \end{cases} \qquad \Rightarrow \qquad \displaystyle \int_{-\infty}^{\infty} \delta(x) f(x) dx = f(0)\]
			tiene una transformada igual una se\~nal arm\'onica, que no est\'a definida en frecuencia espacial (vale lo mismo para tiempo o haciendo la antitransformada). La relaci\'on general que se encuentra es
			\begin{equation}
				\Delta x \Delta k \geq \frac{1}{2}
				\label{eq:ondas_paquetes_incerteza}
			\end{equation}
			que determina la dispersi\'on espacial (o temporal) conociendo la dispresi\'on en frecuencia (o ancho de banda) y visceversa. Es una relaci\'on fundamental que mantiene la informaci\'on entre transformaciones hechas.

\end{document}
