\documentclass[a4paper,spanish]{article}


\usepackage[spanish]{babel}
\usepackage[latin1]{inputenc}
\usepackage{amsmath}
\usepackage{amssymb}
\usepackage[margin=1.5cm]{geometry}
\usepackage{graphicx}
\usepackage{caption}
\usepackage{subcaption}
\usepackage{float}
\newcommand{\oiint}{\displaystyle\bigcirc\!\!\!\!\!\!\!\!\int\!\!\!\!\!\int}


\usepackage{epsfig}
\usepackage{color}
\usepackage{amsfonts}
\usepackage[T1]{fontenc}

\def\Fou {\mathcal{F}}
\def\Rea {\mathcal{R}e}
\def\Ima {\mathcal{I}m}
\def\N {\mathbb{N}}
\def\C {\mathbb{C}}
\def\Q {\mathbb{Q}}
\def\R {\mathbb{R}}
\def\Z {\mathbb{Z}}


%\renewcommand{\contentsname}{\'Indice}
%\renewcommand{\chaptername}{Cap\'\i tulo}
%\renewcommand{\bibname}{Referencias}

\newtheorem{prop}{Proposici\'on}[section]
\newtheorem{teo}[prop]{Teorema}
\newtheorem{defi}[prop]{Definici\'on}
\newtheorem{obs}[prop]{Observaci\'on}
\newtheorem{cor}[prop]{Corolario}
\newtheorem{lema}[prop]{Lema}
\newtheorem{ejem}[prop]{Ejemplo}
\newtheorem{ejer}[prop]{Ejercicio}

\numberwithin{equation}{section}
\newtheorem{definition}{Definicion}


\newenvironment{proof}{
\trivlist \item[\hskip \labelsep\mbox{\it Demostraci\'on:
}]}{\hfill\mbox{$\square$}
%\trivlist \item[\hskip \labelsep{\sl
%#1}\mbox{Demostraci\'on}]}{\hfill\mbox{$\square$}
\endtrivlist}

%\topmargin 0cm \oddsidemargin 0.7cm %% margenes
%\textheight 21cm \textwidth 15cm %% tamaño del texto
\parindent 0cm %% sangria


\begin{document}

\part{Movimiento arm\'onico con muchos grados de libertad}

En este apartado vamos a estudiar los sistemas con varios grados de libertad, enlazados con relaciones lineales, por lo que vale el \textbf{principio de superposici\'on}. (\textbf{Recordatorio:} Dado un sistema lineal con coeficientes constantes, si $\psi_1$ y $\psi_2$ son soluciones de $\dot{\overrightarrow{X}}=\textbf{A}\overrightarrow{X}$, entonces $\psi_1 + \psi_2$ son soluci\'on). Como el sistema es lineal, la ecuaci\'on que vamos a resolver la podemos condensar en la siguiente relaci\'on (en coordenadas generalizadas)
    \begin{equation}
        \ddot{\vec{\psi}} = \sum_i C_i \psi_i = C_i \psi_i = A \vec{\psi}
        \label{eq:oscilador}
    \end{equation}
	y como son lineales tienen como soluciones una superposici\'on o suma de soluciones arm\'onicas, que denominamos modos normales
	\begin{equation}
		\vec{\psi}(t) = \sum_i c_i v_i e^{i (\omega_i t + \phi_i)} =  \sum_i c_i v_i \sen(\omega t + \phi_i)
		\label{eq:oscilador_solucion}
	\end{equation}
	es decir un modo normal es una soluci\'on donde todas las partes se mueven a una frecuencia id\'entica, con fase inicial igual u opuesta a la inmediatamente contigua, pero con una relaci\'on de amplitud caracter\'istica de cada modo. Para obtener la soluci\'on anterior tenemos que probar la soluci\'on $\vec{\psi}(t) = \textbf{c} e^{i \omega t}$, con lo que obtenemos
	\begin{equation}
		\omega^2 \textbf{c} = A \textbf{c}
		\label{eq:oscilador_autovalores}
	\end{equation}
	que es una ecuaci\'on que nos representa los \textit{autovalores} de la matriz del problema $\textbf{A}$ con autovectores $\textbf{c}$, que van a indicar la relaci\'on de amplitudes para cada part\'icula en cada modo normal; y claramente podemos resolverla como aprendimos en \'algebra lineal. Las coordenadas que dejan diagonal al la matriz $A$, es decir que desacopla el sistema, se denominan coordenadas normales, y a veces es posible por simple inspecci\'on obtenerlas (resolviendo el problema m\'as f\'acilmente).

\section{Superposici\'on de vibraciones}
En general
\section{Pulsaciones}

Consideremos un sistema con dos modos de oscilaci\'on con frecuencias muy parecidas (aunque tiene validez exacta la f\'ormula posterior, no tiene mucha importancia f\'isica si no son parecidas) con sus partes en un caso general de oscilaci\'on, es decir que el movimiento es suma de ambos modos. En ese caso definimos la frecuencia de modulaci\'on como
		\begin{equation}
			\omega_{mod} = \frac{|\omega_1 - \omega_2|}{2}
			\label{eq:oscilador_pulsaciones_modulacion}
		\end{equation}
		y la frecuencia de portadora o promedio
		\begin{equation}
			\omega_p = \frac{\omega_1 + \omega_2}{2}
			\label{eq:oscilador_pulsaciones_portadora}
		\end{equation}
		Con esas dos frecuencias podemos reescribir de la siguiente forma el movimiento de cada parte (consideramos que la fase y la amplitud son iguales, en caso contrario simplemente cambia el valor de la amplitud, que pasa a ser $A_1 + A_2$)
		\begin{equation}
			\psi = A_{mod}(t) \sen(\omega_p t) = 2 A \sen(\omega_{mod} t) \sen(\omega_p t)
			\label{eq:oscilador_pulsaciones}
		\end{equation}
		La formula anterior tiene validez para cualquier suma de dos se\~nales arm\'onicas muy cercanas, y para darnos una idea de la forma
		
\section{Oscilaciones de sistemas de muchos grados de libertad}
	  Al ir agrandando la cantidad de grados de libertad es necesario tener otro m\'etodo para describir dicho sistema. El sistema general que usamos para describir el problema es
	  \begin{equation}
		\ddot{\psi_n} = c_{n-1} \psi_{n-1} + c_{n+1} \psi_{n+1} + c_n \psi_n
		\label{eq:oscilador_recurrencia}
	  \end{equation}
	  es decir lo definimos como una relaci\'on de recurrencia, independiente del elemento m\'ovil elegido (se considera que todas las masas y los resortes son iguales, por simplicidad). Proponiendo como soluci\'on
	  \begin{equation}
		 \psi_n(t) = A_n \cos(\omega t - \phi)
		 \label{eq:oscilador_recurrencia_solucion}
	  \end{equation}
	  obtenemos una relaci\'on de recurrencia para las amplitudes
	  \begin{equation}
		 \omega^2 + c_n = \frac{c_{n-1} A_{n-1} + c_{n+1} A_{n+1}}{-A_n}
		 \label{eq:oscilador_recurrencia_relacion}
	   \end{equation}
	   Para la cual proponemos $A_n = \cos(k x_n) = \Re(e^{i k x_n})$, siendo $k = \frac{2 \pi}{\lambda}$ la frecuencia espacial o n\'umero de onda (y $\lambda$ la longitud de onda), obteniendo
	   \begin{equation}
			\frac{c_{n+1} e^{i k x_{n+1}} + c_{n-1} e^{i k x_{n-1}}}{-e^{i k x_n}} = \omega^2 + c_n
			\label{eq:oscilador_recurrencia_rel_dispersion}
	   \end{equation}
	   que es una relaci\'on entre la frecuencia temporal y la frecuencia espacial, lo que se denomina relaci\'on de dispersi\'on. 
	   Luego hay que aplicarle condiciones de contorno al problema, es decir cuanto vale la coordenada $\psi_0$ y $\psi_N$ (siendo $N$ la cantidad de elementos), obteniendo el valor de frecuencia espacial y por lo tanto los modos de oscilaci\'on (frecuencias temporales) posibles.
 \end{document}