%Setea los caracteres extraños para listing
\lstset{literate=
  {á}{{\'a}}1 {é}{{\'e}}1 {í}{{\'i}}1 {ó}{{\'o}}1 {ú}{{\'u}}1
  {Á}{{\'A}}1 {É}{{\'E}}1 {Í}{{\'I}}1 {Ó}{{\'O}}1 {Ú}{{\'U}}1
  {à}{{\`a}}1 {è}{{\`e}}1 {ì}{{\`i}}1 {ò}{{\`o}}1 {ù}{{\`u}}1
  {À}{{\`A}}1 {È}{{\'E}}1 {Ì}{{\`I}}1 {Ò}{{\`O}}1 {Ù}{{\`U}}1
  {ä}{{\"a}}1 {ë}{{\"e}}1 {ï}{{\"i}}1 {ö}{{\"o}}1 {ü}{{\"u}}1
  {Ä}{{\"A}}1 {Ë}{{\"E}}1 {Ï}{{\"I}}1 {Ö}{{\"O}}1 {Ü}{{\"U}}1
  {â}{{\^a}}1 {ê}{{\^e}}1 {î}{{\^i}}1 {ô}{{\^o}}1 {û}{{\^u}}1
  {Â}{{\^A}}1 {Ê}{{\^E}}1 {Î}{{\^I}}1 {Ô}{{\^O}}1 {Û}{{\^U}}1
  {œ}{{\oe}}1 {Œ}{{\OE}}1 {æ}{{\ae}}1 {Æ}{{\AE}}1 {ß}{{\ss}}1
  {ç}{{\c c}}1 {Ç}{{\c C}}1 {ø}{{\o}}1 {å}{{\r a}}1 {Å}{{\r A}}1
  {€}{{\EUR}}1 {£}{{\pounds}}1 {ñ}{{\~n}}1 {Ñ}{{\~N}}1
}

%% Coloreado para Julia
\lstdefinelanguage{Julia}%
  {morekeywords={abstract,break,case,catch,const,continue,do,else,elseif,%
      end,export,false,for,function,immutable,import,importall,if,in,%
      macro,module,otherwise,quote,return,switch,true,try,type,typealias,%
      using,while},%
   sensitive=true,%
   alsoother={$},%
   morecomment=[l]\#,%
   morecomment=[n]{\#=}{=\#},%
   morestring=[s]{"}{"},%
   morestring=[m]{'}{'},%
}[keywords,comments,strings]%

\lstset{%
    language         = Julia,
    basicstyle       = \ttfamily,
    keywordstyle     = \bfseries\color{blue},
    stringstyle      = \color{magenta},
    commentstyle     = \color{ForestGreen},
    showstringspaces = false,
}
{\footnotesize
\begin{lstlisting}[inputencoding=utf8,extendedchars=true]
import PyPlot #ploteo de datos
using LaTeXStrings #Permite usar LaTeX en los strings

#Funciones auxiliares. 
#Nota: Los arrays son pasados por referencia.
#Entonces, no es necesario retornarlos con return

#Función que calcula el cambio de energía al cambiar el spin i
# deltaE =  2 * i * (abajo + arriba + izquierda + derecha)
function deltaE(latt, i)
    L = size(latt)[1]
    N = L * L
    #Condiciones de contorno
    i_down = i % L != 0 ? i + 1 : i - L + 1
    i_up = (i - 1) % L != 0 ? i - 1 : i + L - 1
    i_izq = (i - L) > 0 ? i - L : i - L + N
    i_der = (i + L) <= N ? i + L : i + L - N
    result = latt[i_up] 
    result += latt[i_down]
    result += latt[i_izq] 
    result += latt[i_der]
    result *= 2 * latt[i] 
    return result
end

#Función que implementa la transición entre configuraciones.
#Se ejecuta en cada paso de Monte Carlo y recorre N spines al azar o ordenado
function transicion(latt, T, isRandom = false)
    srand(time_ns())
    L = size(latt)[1] #Tamaño de la red
    N = L * L 
    beta = 1.0 / T
    p = [exp(-beta * 4), exp(-beta * 8)]
    for s = 1:N
            #Permite elegir un spin al azar o ordenadamente
            if isRandom
                i = rand(1:L)
            else
                i = s
            end
            dE = deltaE(latt, i)
            if dE != 0 #No hacer nada si dE = 0
                if dE < 0 || p[dE == 4 ? 1 : 2] > rand()
                    latt[i] *= -1                
                end
            end
        end
end

#Función que inicializa la red de L x L. 
#De forma aleatoria o ordenada
function initLatt(L, isRandom = true)
    
     if isRandom
        #Devuelve un número aleatorio de -1:2:1 que es -1,1
        return rand(-1:2:1, L, L)
        #También podemos usar sign(rand(Int32, L, L))
    else
        #Devuelve red ordenada con unos.
        return ones(L, L)
    end
end

#Funcion utilizada para medir observables
function sampleo(latt)
    
    L = size(latt)[1]
    N = L * L
    m = 0.0
    e = 0.0
    m = sum(latt) #Con sumar es suficiente
    for i = 1:N
        #Multiplica con los vecinos posteriores.
        i_der = (i + L) <= N ? i + L : i + L - N
        i_down = i % L != 0 ? i + 1 : i - L + 1
        e += latt[i] * (latt[i_der] + latt[i_down])
    end
    m /= N
    e /= N
    return (abs(m), e) #Tomo sólo la magnitud de la magnetización
end

#Función donde se implementa la iteración Monte Carlo
function isingMetropolis(L, T, fMed = 100, nSamp = 1000, nTerm = 500, 
                        lattRandom = false, randTrans = false) 
    #Cantidad de iteraciones
    nIter = nSamp * fMed  
    N = L*L   
    latt = initLatt(L, lattRandom) #Inicializa la red
    #Observables
    e = zeros(nSamp + 1)
    mag = zeros(nSamp + 1)
    nSamp = 1 #Resteo el nSamp así lo uso como un indice para los observables
    #Termalizado de la red
    for s = 1:nTerm
        #En cada iteración efecutúa una transición posible
        transicion(latt, T, randTrans)
    end
    #Primera medición
    mag[nSamp], e[nSamp] = sampleo(latt)
    nSamp += 1
    for s = 1:nIter
        #En cada iteración efecutúa una transición posible
        transicion(latt,T, randTrans)
        if s % fMed == 0 #Si el paso es múltiplo de fMed => Mide los observables
            mag[nSamp], e[nSamp] = sampleo(latt)
            nSamp += 1
        end
    end
    return (T, mag, e) #Devuelve una tupla con todos los observables
end
\end{lstlisting}
}





