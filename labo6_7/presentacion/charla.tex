\documentclass{article}
\usepackage[utf8]{inputenc}
\usepackage[spanish]{babel}
\usepackage[margin=1.3cm]{geometry}
\usepackage{url}

\title{labo6_7_plan}
\author{Sebastián Schiavinato}
\date{July 2015}

\begin{document}
{\noindent \huge \textbf{Charla de avance Laboratorio 6 y 7.}}\\
{\Large 2do cuatrimestre 2015}\\
\\
\noindent \textbf{Estudiante:} Sebastián Schiavianto\\
\textbf{Director:} Dr. Hernán Edgardo Grecco\\
\textbf{Codirector:} Dra. Andrea Verónica Bragas\\
\textbf{Lugar de Trabajo:} Laboratorio de Electrónica Cuántica - Departamento de Física - FCEyN - UBA

Salvo que no se ponga numeración, el número de sección está relacionado con una diapositiva. Tal vez alguna sección tenga más para contener los datos prsentados

\section{Introducción}

Buen día, mi nombre es Sebastián Schiavinato y voy a mostrarles el avance mi proyecto de laboratorio 6, que se está desarrollando en el Laboratorio de Electrónica Cuántica (LEC) del Departamento de Física de esta facultad. Este trabajo, con dirección del Dr. Hernan Grecco y codirección de la Dra. Andrea Bragas, tiene como objetivo el diseño y construcción un perfilador de haz portátil y de fácil reprodución, para ser usado en cualquier setup óptico del laboratorio. También tiene como objetivo la construcción de un espectrómetro portátil para aplicaciones del LEC.


\section{Motivaciones}
La motivación de este proyecto gira alrededor de la necesidad de calibrar los haces por medio de alineación y otros procesos a la hora al modificar cualquier experimento óptico. Este proceso, y ahí radica la mayor motivación, conlleva en general muchisimo tiempo y eventualmente es necesario desarmar parte del setup.


Para caracterizar un haz de luz es necesario medir
\begin{itemize}
\item Perfil espacial. Divergencia
\item Perfil espectral y temporal
\item Polarización
\end{itemize}

Nosotros vamos a atacar la medición del perfil espacial (y la divergencia) y el perfil espectral.

\section{Perfilador}

Un perfilador de haz corresponde a un dispositivo capaz de medir la forma espacial en un plano. Si se mueve de plano se puede deducir la divergencia del haz. 

El perfilador usual, el que se construye en Laboratorio 5 en la práctica de LASER, va cortando el haz con un filo e integra la intensidad del haz obturado con un fotodiodo o sensor de potencia (y eventualmente una lente si el haz es muy grande para el sensor); permite usar sensores mucho más baratos, con poco rango dinámico y mucho menos sensibilidad espacial.

Para tener noción de lo que estoy buscando reproducir, podemos ver el perfil de intensidades, normalizado, obtenido por un perfilador manual, para un haz gaussiano (que representa la mayoría de los haces, el TEM00). La función que vemos se denomina la función error, con la propiedad de ser la integral del la función gaussiana.

Con ese método de medición en miras, se consideró cambiar a un sistema automático de corte del haz, por medio de un tambor giratorio accionado por un motor eléctrico. Esta idea ya tiene implementaciones comerciales, por lo que al priori parece estamos bien encaminado.

Ahora voy a pasar a detallar las diferentes partes de perfilador estudiadas y desarrolladas

\section{Diseño mecánico}

Para empezar, el diseño mecánico consistió en pensar en un soporte para el tambor que sea fácil de colocar, portátil y que permita el uso del instrumental en la mayor cantidad de situaciones. También consistió en diseñar el tambor, considerando el tamaño de los haces, y conseguir el motor adecuado para la aplicación.

Como premisas, el diseño del instrumental debe permitir ser colocado en el sistema Cage de ThorLabs, visible en la figura, a 50mm de la mesa, y el tambor debe girar a una velocidad que permita obtener una actulización del perfil en tiempo real.

Con estas premisas, Lo primero que se consideró fue atacar la forma del tambor. 

La forma del tambor consistió desde el principio en un cilindro con una proyección axial, como se ve en la figura, que permite obturar un haz, y obtener el perfil, de hasta 10mm de diámetro (más grande del necesario en el LEC) al girarlo respecto al eje del cilindro. Si el haz tuviera una divergencia apreciable, este diseño permitiría deducirla a partir de la obturaciones en una vuelta entera. La implementación del prototipo de este tambor se hizo con la impresora 3D de plástico del departamento.

La velocidad de tambor, es decir la velocidad, está limitada por la cantidad de muestras que se requieran para la aplicación, es decir permitir alinear en tiempo real, y la velocidad de adquisición de la electrónica. Para alinear en tiempo real, es necesario tener 24 medidas del perfil cada 1 s (que corresponde a la velocidad de reacción del ojo humano), por lo que se necesita elegir un motor capaz de moverse a esa velocidad. 

Con esa premisa, se probaron casi todos los tipos de motores de corriente continua del mercado. Era especialmente necesario que el motor se moviese a una velocidad constante, ya que es mucho más fácil medir el tiempo de rotación y no es necesario utilizar ningún sistema de adquisición de posición. Por eso se descartó casi al principio los motores de escobillas, ya que necesitan retroalimentación para un control efectivo. Esto nos deja con servomotores y motores paso a paso. Los servomotores no industriales, con peso menores a 1kg, más rápidos que se consiguen llegan a 2 revoluciones por segundo, por lo que se descartó el uso de estos motores. A partir de eso, se consiguió un motor paso a paso, modelo NEMA 17 con 200 pasos por vuelta, una velocidad máxima de 3000 pasos por segundo, 4kg$\times$cm y un peso de 300g.  Este motor aún así no tiene suficiente velocidad, logrando entre 10 y 15 revoluciones por segundo, pero al no haber otras opciones en el mercado, lo que se decidió avanzar con esta nueva limitación.

Un motor paso a paso corresponde a un motor de continua con un estator dentado con dos o más fases, el cual al cambiar el sentido de la corriente de las bobinas se puede logar mover el rotor solamente un paso (o menos con microstepping), y además tiene la ventaja de bloquearse si el campo magnético rotatorio y el rotor no están en sincronismo, por lo que mantienen la velocidad impuesta.

Finalmente el soporte del motor y tambor, que se ve en la figura del ensamblaje con el sistema Cage a escala, permite ubicar en dos ejes perpendiculares al motor (y medir el perfil en esos ejes), además de ser de portátil y de fácil de ubicar en otros setups ópticos con los agujeros adicionales.


\section{Diseño de la adquisición de datos}
Para adquirir los datos, se consideró inicialmente utilizar microcontroladores, que nos permité al mismo tiempo mover el motor y adquirir la señal del sensor. En todo el espectro de microcontroladores se eligió la plataforma Arduino, y en particular el microcontrolador Uno, ya que plataforma tiene la ventaja de tener resuelto la programación por medio de un conector USB y ser la plataforma con mayor librerías del mercado. Además tiene una comunidad con muchísima actividad (y por lo tanto casi todos los recovecos de la programación están resueltos).

Lamentablemente al intentar mover el motor ya adquirido y adquirir al mismo tiempo la señal del fotodiodo que el Arduino UNO tiene límites en velocidad (16MHz) y de adquisición de datos analógico (10ksps), y el límite más grande correspondió a la memoria RAM (2KiB) que no permitía guardar una muestra por paso (el mínimo necesario).  Tampoco era viable, por la velocidad, la adquisición en tiempo real por puerto USB, debido a problemas de señalizado y velocidad del microcrontrolador. 

Para solucionar estos problemas se paso a la próxima tecnología compatible con la plataforma de desarrollo Arduino, en este caso un microcontrolador TEENSY versión 3.1; este dispositivo tiene una velocidad de procesador de 96MHz y 64KiB de RAM, tal vez un poco sobredimensionado para la aplicación. Con este microcontrolador llegamos al límite del buffer del USB, así que ya más de aproximadamente 1200 datos (tensión medida y tiempo de medición) no se pudo mandar de forma consistente

Con esto se diseño un circuito (en este caso mide 5x5cm), como se ve en la figura, con el controlador el motor paso a paso, que nos permitio mover el motor hasta 12 revoluciones por segundo, a partir de una aceleración constante, y medir 800 mediciones de la señal del fotodiodo por vuelta. Esto impacta en la capacidad de medir 12 perfiles por segundo en un plano, habiendo dos planos de medición, es decir 24 perfiles en total por segundo.

En este apartado queda terminar de armar una caja con la electrónica dentro y agregar algunas seguridades para el uso diario.

\section{Diseño del análisis de datos}
Finalmente, en la computadora se adquieren los datos, por medio de un programa implementado en Python, se los procesa con un algoritmo de filtrado de señales propio y se obtiene automáticamente el perfil del haz. El algoritmo consiste en encontrar los puntos más cercanos al valor medio de una señal como se ve en la figura, y a partir de ahí recortar partes de la señal. Luego se la ajusta con una función error, que corresponde a intensidad de un haz gaussiano como ya vimos. 

Acá se ve el ajuste de datos con el tamaño del haz en el gráfico, inferido, con su error, de la velocidad del motor y la geometría del tambor. La diferencia en el tamaño es producto de un problema de precesión del tambor, que se espera que se elimine o se reduzca de forma importante al mecanizarlo en metal. Esto es lo único que nos queda resolver el perfilador.

El programa de adquisición no necesita mucho más trabajo, tal vez un poco de presentación y documentación, pero ya está completo en funcionalidad.


\section{Proyecto SOMA}

Ahora, uno de los aspectos de este proyecto que más me motivan y más me gustan de este trabajo, es que está enmarcado en el proyecto SOMA, de Sistemas de OptoMecánica Abierta, y como su nombre lo dice es un proyecto abierto. Es decir que cada parte del perfilador puede usarse, mejorarse y adaptarse sin cargo por cualquier laboratorio o persona. También esto limito el gasto del proyecto y los métodos de construcción utilizados, ya que se debió pensar para reproducirse facilmente fácilmente.

\section{Objetivos que quedan}

Como ya se mencionó, del perfilador está casi todo resuelto y puede considerarse como cerrado. Uno de los objetivos del proyecto inicial corresponden la construcción de un espectrómetro, con un circuito integrado Hamammatsu C12666MA. Este integrado todavía no está todavía disponible para el laboratorio, pero para diciembre ya va a estar en disposición. Este integrado va a utilizar un Arduino UNO, para el cual ya está resuelta la adquisición, y algún adaptador optomecánico para las fuente.

\section{Objetivos de Laboratorio 7}
Para finalizar, estos son los proyectos en Laboratorio 7, los cuales van a usar extensivamente el ambos instrumentales:
\begin{itemize}
\item Caracterización del espectro de emisión de nanolamparas de Lantánidos (experiencia efectuada el marco del trabajo doctoral del Ing. Bujjamer, con financiamiento de la Fundación Argentina de Nanotecnología)
\item Mejora de la resolución del SPIM mediante la optimización del perfil del haz (trabajo doctoral del Lic. Moretti, financiado por ANPCyT PICT-2013-1301)
\end{itemize}

Eventualmente habrá otras ideas dadas las especificaciones de los equipos finales.


\end{document}